% Indicate the main file. Must go at the beginning of the file.
% !TEX root = ../main.tex

\chapter{Test Results}
\label{Chapter5}

This chapter presents the results of the test procedures conducted to analyze KEPLER-produced metrics. Each section corresponds to a specific resource type—CPU, Memory, Disk I/O, and Network I/O—with further division into container-level and node-level metrics. The results are discussed alongside their respective figures, which illustrate KEPLER-deduced power consumption and performance trends. 

It is important to note that all KEPLER metrics exhibit high oscillations. A detailed analysis reveals that these oscillations follow a highly regular pattern, suggesting an issue with the metric publication intervals in KEPLER or the scraping intervals of Prometheus. The data has been analyzed as-is, with moving averages provided to improve readability. The implications of this irregularity will be discussed further in the Discussion chapter.

For readability and to avoid confusion, two KEPLER metric concepts should be reiterated before discussing the results:
\begin{itemize}
    \item \textbf{\textit{Package} Metrics:} These metrics contain the entire CPU package, including all cores and uncore components.
    \item \textbf{\textit{Platform} Metrics:} These metrics the entire node.
    \item \textbf{\textit{Other} Metrics:} These metrics contain the entire platform except the CPU package and DRAM.
\end{itemize}

\section{CPU Stress Test Results}
\subsection{Container-Level CPU Metrics}

A set of figures illustrating cache misses, CPU cycles, and CPU instructions during testing is provided in Figures~\ref{fig:cpu_kepler_container_cache_miss_total} to ~\ref{fig:cpu_kepler_container_cpu_cycles_total}.

\begin{figure}[H]
    \centering
    \begin{subfigure}{0.49\textwidth}
        \includegraphics[width=\textwidth]{Figures/diagrams/cpu/kepler_container_cache_miss_total/cpu_kepler_container_cache_miss_total_smoothed.png}
        \caption{Cache misses}
        \label{fig:cpu_kepler_container_cache_miss_total}
    \end{subfigure}
    \begin{subfigure}{0.49\textwidth}
        \includegraphics[width=\textwidth]{Figures/diagrams/cpu/kepler_container_cpu_instructions_total/cpu_kepler_container_cpu_instructions_total_smoothed.png}
        \caption{CPU instructions}
        \label{fig:cpu_kepler_container_cpu_instructions_total}
    \end{subfigure}
    \begin{subfigure}{1\textwidth}
        \includegraphics[width=\textwidth]{Figures/diagrams/cpu/kepler_container_cpu_cycles_total/cpu_kepler_container_cpu_cycles_total_smoothed.png}
        \caption{CPU cycles}
        \label{fig:cpu_kepler_container_cpu_cycles_total}
    \end{subfigure}
    \caption[Container-Level CPU Metrics]{KEPLER container-level CPU metrics during a CPU stress test}
\end{figure}

These three figures show the cache misses, CPU cycles, and CPU instructions of the test container during testing. Notably, the diagrams are very uniform, as the three metrics directly mirror the work generated by stress-ng, which is expected to be even and uniform. The direct correlation between test workload and cache misses, CPU cycles, and CPU instructions is very clean, confirming the correct execution of the workload.

Since the CPU workload running on the rest of the cluster should not affect the test container’s workload, the metric values in the 'idle' and 'busy' cluster conditions should be the same. This is indeed the case, verifying that testing was conducted correctly.

A figure illustrating KEPLER's deduced package power consumption is provided in Figure~\ref{fig:cpu_kepler_container_package_joules_total}.

\begin{figure}[H]
    \centering
    \includegraphics[width=1\textwidth]{Figures/diagrams/cpu/kepler_container_package_joules_total/cpu_kepler_container_package_joules_total_smoothed.png}
    \caption[Container Package Power]{KEPLER container-level package power consumption.}
    \label{fig:cpu_kepler_container_package_joules_total}
\end{figure}

The figure shows a clear upward trend in package power consumption, with distinct steps that correspond to the expected workload increases. A strong correlation is observed between KEPLER's reported package power consumption and the test workload. However, the relationship between power consumption and workload is non-linear: while a 10\% workload averages around 2.5 Watts, a 90\% workload results in only a doubling of power consumption despite the workload increasing by a factor of 9.

Additionally, KEPLER's package power measurements remain consistent regardless of whether the node is idle or busy, showing no statistically significant difference.

A set of figures illustrating container Joules, DRAM Joules, and other power consumption components is provided in Figures ~\ref{fig:cpu_kepler_container_joules_total} to ~\ref{fig:cpu_kepler_container_dram_joules_total}.

\begin{figure}[H]
    \centering
    \begin{subfigure}{1\textwidth}
        \includegraphics[width=\textwidth]{Figures/diagrams/cpu/kepler_container_joules_total/cpu_kepler_container_joules_total_smoothed.png}
        \caption{Total container-level energy consumption.}
        \label{fig:cpu_kepler_container_joules_total}
    \end{subfigure}
    \begin{subfigure}{0.49\textwidth}
        \includegraphics[width=\textwidth]{Figures/diagrams/cpu/kepler_container_other_joules_total/cpu_kepler_container_other_joules_total_smoothed.png}
        \caption{Non-CPU/DRAM energy consumption}
        \label{fig:cpu_kepler_container_other_joules_total}
    \end{subfigure}
    \begin{subfigure}{0.49\textwidth}
        \includegraphics[width=\textwidth]{Figures/diagrams/cpu/kepler_container_dram_joules_total/cpu_kepler_container_dram_joules_total_smoothed.png}
        \caption{DRAM energy consumption.}
        \label{fig:cpu_kepler_container_dram_joules_total}
    \end{subfigure}
    \caption[Container-Level Energy Consumption]{KEPLER container-level energy consumption during a CPU stress test}
\end{figure}

The three figures for container Joules, DRAM Joules, and other power components (representing host components excluding CPU and DRAM) show less direct correlation with workload than the package power. 

Observations:
\begin{itemize}
\item In figure ~\ref{fig:cpu_kepler_container_joules_total}, a slight upwards trend is detectable as the measured power consumption of the entire node rises by about 5 Watts. This change mirrors the increase in package power consumption from the previous figure ~\ref{fig:cpu_kepler_container_package_joules_total}, where an approximately 5 Watt increase can be seen.
\item In figure ~\ref{fig:cpu_kepler_container_other_joules_total}, showing the "Other" (i.e. Non-CPU/DRAM) container energy consumption, no trend can be seen in the data. This is to be expected, since only the CPU itself was stressed. However, the high amount of the Non-CPU/DRAM power consumption surprises, given that it is roughly double the amount of the CPU Package Energy consumption.
\item The measured DRAM is not affected by CPU stress, which is as expected. With between 0.5 and 1 Watt, DRAM energy consumption is comparably low.
\item During the second part of the experiment (i.e during testing with a busy node), all metrics seem to have a slightly more smooth curve.
\end{itemize}

\subsection{Node-Level CPU Metrics}

Figures illustrating node-level DRAM Joules, other component Joules, and package Joules are provided in Figure~\ref{fig:cpu_kepler_node_package_joules_total}.

\begin{figure}[H]
    \centering
    \begin{subfigure}{1\textwidth}
        \includegraphics[width=\textwidth]{Figures/diagrams/cpu/kepler_node_package_joules_total/cpu_kepler_node_package_joules_total_ho3.png}
        \caption{Node Package energy consumption}
        \label{fig:cpu_kepler_node_package_joules_total}
    \end{subfigure}
    \begin{subfigure}{0.49\textwidth}
        \includegraphics[width=\textwidth]{Figures/diagrams/cpu/kepler_node_dram_joules_total/cpu_kepler_node_dram_joules_total_ho3.png}
        \caption{Node DRAMenergy consumption}
        \label{fig:cpu_kepler_node_dram_joules_total}
    \end{subfigure}
    \begin{subfigure}{0.49\textwidth}
        \includegraphics[width=\textwidth]{Figures/diagrams/cpu/kepler_node_other_joules_total/cpu_kepler_node_other_joules_total_ho3.png}
        \caption{Node Non-CPU/DRAM energy consumption}
        \label{fig:cpu_kepler_node_other_joules_total}
    \end{subfigure}
    \caption[Node-Level Energy Consumption]{KEPLER node-level energy consumption during a CPU stress test}
\end{figure}


In figures ~\ref{fig:cpu_kepler_node_package_joules_total} to ~\ref{fig:cpu_kepler_node_other_joules_total}, the total node power consumption is shown. At a node level, KEPLER distinugishes between idle and dynamic power consumption.

Observations:
\begin{itemize}
\item The most strinking obervation is the relatively high idle power consumption of the node, seen in all figures. While figure ~\ref{fig:cpu_kepler_node_package_joules_total} shows the rising dynamic power consumption resulting from the CPU stress test, the idle power consumption still far exceeds the dynamic power.
\item A key deduction is that the dynamic "Other"'-power consumption seen in figure ~\ref{fig:cpu_kepler_node_other_joules_total} is independent from the CPU stress test load. This further supports the conclusion that while "Other"' system components contribute most to the overall platform energy consumption, the are generally unaffected by CPU workload.
\end{itemize}

\subsubsection{Overall Conclusions}
The CPU stress test results demonstrate that KEPLER accurately captures workload-dependent variations in power consumption. Key takeaways from the analysis include:
\begin{itemize}
    \item KEPLER's CPU package power measurements exhibit a correlation with workload intensity, although the relationship is non-linear.
    \item High idle power consumption at the node level suggests that a significant portion of energy use is independent of CPU workload.
    \item The "Other" component power consumption remains largely static, reinforcing that non-CPU/DRAM components contribute primarily to baseline energy consumption rather than dynamic variations.
\end{itemize}


% General obeservations: 
% - energy consumption at 90\% load is never 9 times the energy consumption at 10\% load
% - Node power: the idle energy consumption is incredibly high
%     check SPECpower, SERT
%     KUBERNETES Overhead? Server overhead?
% - There seems to be quite a high oscillation of power all KEPLER metrics, irrespective of testing load or tool
%     issue with scraping?
%     issue with prometheus?
    

% CPU figures
% thesis/Figures/diagrams/cpu/kepler_container_cache_miss_total/cpu_kepler_container_cache_miss_total_smoothed.png


\section{Memory}

\begin{figure}[H]
    \centering
    \begin{subfigure}{0.49\textwidth}
        \includegraphics[width=\textwidth]{Figures/diagrams/mem/kepler_container_dram_joules_total/mem_kepler_container_dram_joules_total_smoothed.png}
        \caption{DRAM energy consumption.}
    \end{subfigure}
    \begin{subfigure}{0.49\textwidth}
        \includegraphics[width=\textwidth]{Figures/diagrams/mem/kepler_container_package_joules_total/mem_kepler_container_package_joules_total_smoothed.png}
        \caption{Package energy consumption.}
    \end{subfigure}
    \begin{subfigure}{0.49\textwidth}
        \includegraphics[width=\textwidth]{Figures/diagrams/mem/kepler_container_other_joules_total/mem_kepler_container_other_joules_total_smoothed.png}
        \caption{Non-CPU/DRAM energy consumption.}
    \end{subfigure}
    \begin{subfigure}{0.49\textwidth}
        \includegraphics[width=\textwidth]{Figures/diagrams/mem/kepler_container_joules_total/mem_kepler_container_joules_total_smoothed.png}
        \caption{Total KEPLER container energy consumption.}
    \end{subfigure}
    \caption{Container-level energy consumption during a memory stress test.}
\end{figure}

Placeholder

\begin{figure}[H]
    \centering
    \begin{subfigure}{0.49\textwidth}
        \includegraphics[width=\textwidth]{Figures/diagrams/mem/kepler_node_dram_joules_total/mem_kepler_node_dram_joules_total_ho2.png}
        \caption{DRAM energy consumption (worker node 1).}
    \end{subfigure}
    \begin{subfigure}{0.49\textwidth}
        \includegraphics[width=\textwidth]{Figures/diagrams/mem/kepler_node_dram_joules_total/mem_kepler_node_dram_joules_total_ho3.png}
        \caption{DRAM energy consumption (worker node 2).}
    \end{subfigure}
    \caption{Node-level DRAM energy consumption.}
\end{figure}

Placeholder

\begin{figure}[H]
    \centering
    \begin{subfigure}{0.49\textwidth}
        \includegraphics[width=\textwidth]{Figures/diagrams/mem/kepler_node_other_joules_total/mem_kepler_node_other_joules_total_ho2.png}
        \caption{Non-CPU/DRAM energy consumption}
    \end{subfigure}
    \begin{subfigure}{0.49\textwidth}
        \includegraphics[width=\textwidth]{Figures/diagrams/mem/kepler_node_package_joules_total/mem_kepler_node_package_joules_total_ho2.png}
        \caption{Package energy consumption.}
    \end{subfigure}
    \caption{Node-level memory energy consumption during a memory stress test}
\end{figure}

Placeholder

\section{Disk I/O}

\begin{figure}[H]
    \centering
    \includegraphics[width=1\textwidth]{Figures/diagrams/diskIO/kepler_container_bpf_block_irq_total/diskIO_kepler_container_bpf_block_irq_total_smoothed.png}
    \caption{Block IRQ Metrics during a disk I/O stress test}
\end{figure}

Placeholder

\begin{figure}[H]
    \centering
    \begin{subfigure}{1\textwidth}
        \includegraphics[width=\textwidth]{Figures/diagrams/diskIO/kepler_container_cache_miss_total/diskIO_kepler_container_cache_miss_total_smoothed.png}
        \caption{Cache misses}
    \end{subfigure}
    \begin{subfigure}{0.49\textwidth}
        \includegraphics[width=\textwidth]{Figures/diagrams/diskIO/kepler_container_cpu_cycles_total/diskIO_kepler_container_cpu_cycles_total_smoothed.png}
        \caption{CPU instructions}
    \end{subfigure}
    \begin{subfigure}{0.49\textwidth}
        \includegraphics[width=\textwidth]{Figures/diagrams/diskIO/kepler_container_cpu_instructions_total/diskIO_kepler_container_cpu_instructions_total_smoothed.png}
        \caption{CPU cycles}
    \end{subfigure}
    \caption{Container-level CPU metrics during a Disk I/O stress test}
\end{figure}

Placeholder

\section{Network I/O}

\begin{figure}[H]
    \centering
    \begin{subfigure}{1\textwidth}
        \includegraphics[width=\textwidth]{Figures/diagrams/netIO/kepler_container_cache_miss_total/netIO_kepler_container_cache_miss_total_smoothed.png}
        \caption{Cache misses}
    \end{subfigure}
    \begin{subfigure}{0.49\textwidth}
        \includegraphics[width=\textwidth]{Figures/diagrams/netIO/kepler_container_cpu_cycles_total/netIO_kepler_container_cpu_cycles_total_smoothed.png}
        \caption{CPU instructions}
    \end{subfigure}
    \begin{subfigure}{0.49\textwidth}
        \includegraphics[width=\textwidth]{Figures/diagrams/netIO/kepler_container_cpu_instructions_total/netIO_kepler_container_cpu_instructions_total_smoothed.png}
        \caption{CPU cycles}
    \end{subfigure}
    \caption{Container-level CPU metrics during a network I/P stress test}
\end{figure}

placeholder

\begin{figure}[H]
    \centering
    \begin{subfigure}{0.49\textwidth}
        \includegraphics[width=\textwidth]{Figures/diagrams/netIO/kepler_container_bpf_net_rx_irq_total/netIO_kepler_container_bpf_net_rx_irq_total_smoothed.png}
        \caption{RX IRQ}
    \end{subfigure}
    \begin{subfigure}{0.49\textwidth}
        \includegraphics[width=\textwidth]{Figures/diagrams/netIO/kepler_container_bpf_net_tx_irq_total/netIO_kepler_container_bpf_net_tx_irq_total_smoothed.png}
        \caption{TX IRQ}
    \end{subfigure}
    \caption{Container-level IRQ metrics during a network I/O}
\end{figure}

Placeholder
