\chapter{Introduction and Context} % Main chapter title
\label{Chapter1}

\section{Significance of Energy Efficiency in Cloud Computing}

Cloud computing has revolutionized how computing resources are shared and utilized, offering increased operational efficiency through resource sharing. While economic benefits such as reduced costs and improved scalability are often highlighted, energy efficiency and environmental impact are equally important. By maximizing shared resource utilization, cloud computing inherently reduces energy waste, supporting global sustainability goals.

The rapid adoption of cloud computing has transformed it into a dominant segment of global IT infrastructure. Hyperscalers like Amazon Web Services, Google Cloud, and Microsoft Azure contribute significantly to global energy consumption, attracting attention from policymakers. While cloud providers have incorporated renewable energy sources into their operations, this alone does not address the efficiency of energy utilization for workloads.

Technological advancements have improved the energy efficiency of data centers, with modern facilities achieving Power Usage Effectiveness (PUE) values near 1. However, PUE measures facility-level efficiency, not workload-level efficiency. Even with a perfect PUE of 1, significant energy waste can occur if computational resources are underutilized. This highlights the need to focus on workload-level energy efficiency.

Containers, as lightweight virtualization technology, improve resource density and energy efficiency compared to traditional virtual machines (VMs). Despite these advantages, containers introduce additional complexity, especially in measuring energy consumption. Accurate container-level energy consumption requires granular monitoring tools, a challenge compounded by Kubernetes’ dynamic resource allocation and scaling mechanisms.

Despite the importance of energy efficiency in cloud computing, research on this topic remains limited. While data center operations have been optimized and green coding practices promoted, Kubernetes’ energy efficiency is underexplored. Addressing this gap is crucial for balancing economic and environmental goals.

\section{The Need for Energy-Efficient Kubernetes Clusters}

Kubernetes has become the standard for container orchestration, managing containerized workloads at scale. However, its success brings new challenges, particularly in energy efficiency. Kubernetes environments are complex, featuring dynamic scaling, resource allocation, and workload distribution across multiple nodes. These features, while essential for performance, complicate energy measurement and optimization.

Commonly, Kubernetes clusters are hosted on VMs to simplify infrastructure management, adding another abstraction layer and further complicating energy measurement. To improve energy efficiency, robust methods for measuring energy consumption in Kubernetes environments are necessary. These methods must translate measurements into valuable insights to that will lay the foundation for optimization efforts.

Given the growing energy footprint of cloud computing and the limited research focus on this area, energy-efficient Kubernetes clusters represent a pressing research topic. By addressing this gap, this work contributes to the broader goal of sustainable cloud computing.

\section{Objectives and Scope of this Thesis}

\subsection{Context}

This thesis is part of the Master's program in Computer Science at the Zurich University of Applied Sciences (ZHAW) and represents the first of two specialization projects. The current project (VT1) focuses on the practical implementation of a test environment for energy efficiency research in Kubernetes clusters. The second project (VT2) will explore theoretical aspects and methodologies for measuring and improving energy efficiency.

This thesis builds upon prior works focused on performance optimization and energy measurement. EVA1 covered topics such as operating system tools, statistics, and eBPF, while EVA2 explored energy measurement in computer systems, covering hardware, firmware, and software aspects. These foundational topics provide the basis for the current thesis but will not be revisited in detail.

\subsection{Scope}

This thesis focuses on the practical implementation of a test environment, excluding detailed theoretical analysis and literature reviews. The primary goal is to document the creation of a reliable and reproducible test environment that supports future research on energy efficiency in Kubernetes clusters.

\subsection{Objectives}

The main objective is to design and implement a test environment that facilitates:
\begin{itemize}
\item Analysis of key parameters affecting energy efficiency in Kubernetes clusters.
\item Reliable and consistent experimentation.
\item Reproducibility and automation in deployment and configuration.
\end{itemize}
The outcomes will provide the necessary infrastructure for subsequent research projects.

\subsubsection{Parameters for Analysis}

This project aims to reuse established tools and components where feasible. The parameters to be analyzed include:
\begin{itemize}
\item CPU utilization and energy consumption.
\item Memory usage and its impact on power draw.
\item Disk I/O and storage-related power consumption.
\end{itemize}
Additional parameters may be incorporated based on further evaluation.

\subsubsection{Data Integrity and Persistence}

Ensuring data integrity and persistence is critical for reliable analysis. Key requirements include:
\begin{itemize}
\item Persistent storage that survives system shutdown.
\item A unified data store accessible by all nodes.
\item Data retention across Kubernetes cluster reinstallations.
\item The ability to power down unused worker nodes without data loss.
\end{itemize}

\subsubsection{Reproducibility and Automation}

Reproducibility and automation are bonus goals aimed at enhancing research efficiency. Benefits include:
\begin{itemize}
\item Simplified recovery from misconfiguration through rapid redeployment.
\item Reduced troubleshooting time.
\item Improved stack cleanliness by eliminating residual configurations.
\end{itemize}

\subsubsection{Security}

Security, while not a primary focus, will be addressed by implementing basic best practices. Key security measures include:
\begin{itemize}
\item Use of encrypted passwords.
\item Adherence to basic Kubernetes security best practices.
\item Minimization of potential vulnerabilities through careful configuration.
\end{itemize}

