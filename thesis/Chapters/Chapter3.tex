% Indicate the main file. Must go at the beginning of the file.
% !TEX root = ../main.tex

\chapter{Implementation} % Main chapter title
\label{Chapter3}

This chapter describes the implementation and configuration of the various components used in this project. All automation scripts are designed to be idempotent. All scripts can be executed with shell scripts in the \texttt{Powerstack/scripts}-directory. Generally, all configuration is to be done in the central configuration file (\texttt{/Powerstack/configs/inventory.yml}) unless otherwise stated. Sensitive information is to be defined in the ansible-vault file\\(\texttt{/Powerstack/configs/vault.yml}). To keep this chapter brief, instructions for verification are written in Appendix \ref{AppendixB}

\section{K3s Installation}

This section describes the steps involved in setting up a Kubernetes cluster using K3s on bare-metal servers. The installation was automated using an Ansible playbook forked from the official k3s-io/k3s-ansible \parencite{k3s-ansible} repository, with necessary customizations for internal IP-based communication.
\subsection{Preparing the Nodes}

Before running the Ansible playbook, the following prerequisites need to be in place on all servers:

\begin{itemize}
    \item \textbf{Operating System:} Ubuntu 22.04 (used kernel version 5.15.0)
    \item \textbf{Passwordless SSH:} Passwordless SSH access must be configured for a user with sudo privileges on all servers.
    \item \textbf{Networking:} Each server should have both an internal IP (for cluster communication) and an external IP (for access via VPN or external management).
    \item Local Ansible Control Node Setup:
    \begin{itemize}
        \item \textbf{Ansible-community} 9.2.0 (must be 8.0+).
        \item \textbf{Python} 3.12.3 and Jinja 3.1.2 installed as dependencies.
        \item \textbf{kubectl} 1.31.3
    \end{itemize}
\end{itemize}

\subsection{K3s Installation with Ansible}

The playbook supports x64, arm64, and armhf architectures. For this project, it was tested on x64 architecture only.

\subsubsection{Configuration Details}
\begin{itemize}
    \item Internal and external IP addresses of all servers must be specified.
    \item One server must be designated as the control node.
    \item Default configurations such as ansible-port, ansible-user, and k3s-version can be changed if needed.
\end{itemize}

\subsubsection{Kubectl Configuration}
\begin{itemize}
    \item The playbook automatically sets up kubectl for the user on the Ansible control node by copying the Kubernetes config file from the control node to the local machine.
    \item The user must rename the config file from 'config-new' to 'config' and set the context to powerstack using the following command:\\
    \texttt{kubectl config use-context powerstack}
\end{itemize}

\section{NFS Installation and Setup}

\subsection{NFS Installation with Ansible}

Setting up the NFS server and clients was fully automated using an Ansible playbook. Before beginning the automated setup, the following manual step must be completed:

\begin{itemize}
\item \textbf{Disk Selection:} A suitable disk must be chosen on the control node to act as persistent storage. It is important to note that this disk will be reformatted, and all existing data will be lost.
\end{itemize}

The Ansible playbook performs the following actions:

\begin{itemize}
\item \textbf{Disk Preparation:} The selected disk is partitioned (if necessary) and formatted with a single Btrfs partition. The entire disk space is allocated to this partition. The partition is then mounted to \texttt{/mnt/data}, and an entry is added to \texttt{/etc/fstab} to ensure persistence across reboots.
\item \textbf{NFS Server Setup:} The \texttt{nfs-kernel-server} package is installed and configured on the control node. The directory \texttt{/mnt/data} is exported as an NFS share, accessible to the worker nodes.
\item \textbf{NFS Client Setup:} On each worker node, the \texttt{nfs-common} package is installed. The NFS share is mounted, and an \texttt{/etc/fstab} entry is created to ensure persistence across reboots.
\end{itemize}

\subsubsection{Configuration Details}

\begin{itemize}
    \item The nfs network must be specified, and the control and worker nodes must be in that network
    \item The export path must be specified
\end{itemize}

\section{Rancher Installation and Setup}

\subsection{Rancher Installation with Ansible and Helm}

Although not strictly necessary for the project, Rancher was deployed in the\\
\texttt{cattle-system} namespace to assist with debugging and system analysis. The installation was automated using an Ansible playbook, which integrates Helm for deploying Rancher and its dependencies. The key steps are as follows:

\begin{itemize}
\item \textbf{Helm Installation:} Helm was installed on the control node to facilitate the deployment of Rancher and its dependencies.
\item \textbf{Namespace Creation:} The \texttt{cattle-system} namespace was created to host the Rancher deployment.
\item \textbf{Cert-Manager Deployment:} Cert-Manager, a prerequisite for Rancher, was installed to manage TLS certificates.
\item \textbf{Rancher Deployment:} Rancher was installed using the official Helm chart. During installation, the following parameters were configured:
  \begin{itemize}
  \item \textbf{Hostname:} A Rancher hostname was defined to enable access.
  \item The Helm chart was configured with the \texttt{--set tls=external} option to enable external access to Rancher.
  \item \textbf{Bootstrap Password:} A secure bootstrap password was set for the default Rancher administrator account.
  \end{itemize}
\item \textbf{Ingress Configuration:} An ingress resource was configured to route traffic to Rancher, allowing access through the defined hostname.
\end{itemize}



\section{Monitoring Stack Installation and Setup with Ansible}

The monitoring stack, comprising Prometheus, Grafana, and AlertManager, was deployed using the kube-prometheus-stack\parencite{prometheus_helm_charts}-Helm chart from the\\\texttt{prometheus-community/helm-charts} repository. While the repository was forked for convenience, no changes were made to the upstream chart, ensuring compatibility with future updates.

\subsection{Prometheus and Grafana Installation with Ansible and Helm}

The installation process was automated using Ansible roles, ensuring idempotency and centralization of configurations. The following key steps were executed:

\begin{itemize}  
    \item \textbf{Persistent Storage Configuration:}
    \begin{itemize}
        \item Directories for Prometheus, Grafana, and AlertManager were created on the NFS-mounted disk.
        \item A custom \texttt{StorageClass} was defined for the NFS storage. The default storage \texttt{StorageClass} local-path was overridden to be non-default.
        \item PersistentVolumes (PVs) were created for Prometheus, Grafana, and AlertManager. A PersistentVolumeClaim (PVC) was explicitly created for Grafana, while PVCs for Prometheus and AlertManager were managed by the Helm chart.
    \end{itemize}

    \item \textbf{Helm Chart Installation:}
    \begin{itemize}
        \item A Helm values file was generated dynamically using a Jinja template. This template incorporated variables from the central Ansible configuration file to ensure consistency. Sensitive information, such as the Grafana admin password, was included in the values file. To mitigate potential security risks, the values file was removed from the control node after installation.
        \item The Helm chart was installed using an Ansible playbook. The following customizations were applied via the generated values file:
        \begin{itemize}
            \item PVC sizes for Prometheus and AlertManager were set based on the central configuration.
            \item A Grafana admin password was defined.
            \item Prometheus scrape configurations were adjusted to include the KEPLER endpoints.
            \item Changes to the \texttt{securityContext} were made to allow Prometheus to scrape KEPLER metrics.
        \end{itemize}
    \end{itemize}

    \item \textbf{Service Port Forwarding:}
    \begin{itemize}
        \item Prometheus, Grafana, and AlertManager services were exposed using static \texttt{NodePort}s defined in the central configuration file, enabling external access.
    \end{itemize}

    \item \textbf{Cleanup:}
    \begin{itemize}
        \item A cleanup playbook was executed to remove sensitive configuration files from both the control node and the Ansible control node.
    \end{itemize}
\end{itemize}

\subsection{Removal Playbook}

An Ansible playbook was created to handle the complete uninstallation of the monitoring stack. This was necessary to ensure that PVs and PVCs were explicitly removed to avoid residual artifacts in the Kubernetes cluster.


\section{KEPLER installation and setup with Ansible and Helm}
\subsection{Preparing the environment}
\subsection{KEPLER deployment with Ansible and Helm}
\subsection{Verifying KEPLER metrics}
\subsection{Troubleshooting KEPLER installation}
\subsubsection{Kernel Konfiguration issues}
\subsubsection{Access restrictions}
\subsection{Verification}


