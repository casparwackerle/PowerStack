\chapter{Conclusion and Perspectives}
\label{chap:conclusion-perspectives}

\section{Conclusion}
\label{sec:conclusion}

This thesis set out to examine whether workload-level energy consumption can be
meaningfully measured and attributed in modern Kubernetes environments, despite
heterogeneous hardware, delayed telemetry, and limited observability. Rather than
approaching this problem through prediction or coarse estimation, the work
adopted an accuracy-first perspective, emphasizing explicit accounting,
interpretability, and principled treatment of uncertainty.

Through the design and realization of Tycho, this thesis shows that such an
approach is not only conceptually sound, but practically achievable. While
energy measurement in complex systems remains inherently constrained by the
quality and structure of available telemetry, the results demonstrate that
careful system design can substantially improve the coherence, transparency,
and explanatory value of workload-level energy observations.

By framing energy attribution as a disciplined measurement problem rather than
an exercise in inference, this work reframes what can reasonably be expected
from energy analysis in containerized systems. The thesis thus establishes a
foundation for treating energy as an explicit and interpretable system property,
opening the door to more principled investigation of energy behavior in
Kubernetes-based environments.

\section{Summary of Contributions}
\label{sec:conclusion-contributions}

The central contribution of this thesis is the conception, realization, and
systematic examination of Tycho, an accuracy-first system for workload-level
energy measurement and attribution in Kubernetes environments. Tycho was
designed explicitly as a research instrument, prioritizing semantic clarity,
temporal coherence, and explicit treatment of uncertainty over ease of use or
minimal system requirements.

Beyond the concrete system itself, this work advances a methodological position
on energy measurement in complex, multitasking systems. It demonstrates that
treating energy attribution as a constrained accounting problem, rather than as
an implicit estimation task, enables more interpretable and scientifically
meaningful observations. By enforcing explicit semantics, closed energy
accounting, and delay-aware temporal interpretation, the thesis establishes a
framework in which limitations of telemetry are surfaced rather than obscured.

At the systems level, the thesis shows that heterogeneous and imperfect energy
telemetry can be integrated into a coherent measurement pipeline without
resorting to opaque smoothing or heuristic gap-filling. The implementation of
Tycho illustrates how fine-grained data collection, historical context, and
analysis-driven interpretation can be combined to support workload-level energy
analysis under realistic observability constraints.

Taken together, these contributions demonstrate that accuracy-first energy
measurement is both feasible and valuable as a foundation for research into
energy behavior of containerized workloads. The thesis provides not only a
working system, but also a principled perspective on how energy should be
measured, interpreted, and reasoned about in modern distributed computing
environments.

\section{Perspectives and Future Work}
\label{sec:conclusion-future-work}

Several directions for future work emerge naturally from the design decisions,
limitations, and scope of this thesis. These perspectives do not represent
missing components of the proposed system, but rather extensions that become
relevant once accuracy-first energy measurement is established as a viable and
useful research instrument.

\paragraph{Improved and Alternative Telemetry Sources.}
A central limitation of system-level energy attribution at fine temporal
granularity arises from the characteristics of widely available telemetry
interfaces, in particular coarse resolution and variable delay. While such
interfaces were intentionally used in this thesis to ensure broad deployability
without hardware modifications, future research could explore the integration of
external power measurement devices to obtain higher-quality system energy data.
External meters can provide temporally finer and more stable measurements of
total system power, improving both system-level attribution and the fidelity of
residual energy estimation. Although this approach departs from the constraint of
unmodified server operation, it represents a straightforward extension from an
architectural perspective and has been successfully applied in related research.
Tycho’s design is compatible with such telemetry and would directly benefit from
higher-quality system-level measurements.

\paragraph{Residual Energy Decomposition and Additional Components.}
Tycho deliberately aggregates energy consumption of server components that
cannot currently be measured with sufficient accuracy or granularity into a
residual category. This reflects a principled design choice rather than an
incomplete implementation, as no standardized interfaces currently provide
measurement quality comparable to CPU, memory, or GPU telemetry for components
such as network interfaces, storage devices, or cooling systems. Future work
could investigate the controlled use of proxy metrics to model subsets of
residual energy, accepting reduced semantic guarantees where appropriate. In
addition, the emergence of standardized, fine-grained energy interfaces for
additional components would allow Tycho to extend its attribution scope without
violating its accuracy-first principles.

\paragraph{Quantitative Validation and Comparative Evaluation.}
While this thesis focuses on structural correctness, interpretability, and
qualitative behavior, a comprehensive quantitative validation of workload-level
energy attribution remains an open challenge. Such validation would require
ground-truth measurements across multiple server components simultaneously,
which is difficult to achieve with existing instrumentation. Future work could
pursue partial or staged validation strategies, as well as controlled comparisons
with alternative attribution approaches, to further characterize the accuracy
bounds and limitations of Tycho. Although such efforts would be substantial, they
would provide valuable insight into the strengths and weaknesses of
accuracy-first attribution under different operating conditions.

\paragraph{Integration into Energy-Aware Scheduling and Placement Research.}
Tycho is fundamentally designed as a measurement instrument for research on
workload behavior and resource management in Kubernetes environments. One natural
direction for future work is its use in studies of energy-aware scheduling,
workload placement, and cluster optimization, where fine-grained and
semantically interpretable energy measurements are required to evaluate
scheduling decisions. Rather than prescribing specific control mechanisms or
optimization strategies, Tycho provides the measurement foundation upon which
such research can build. Feedback and requirements emerging from its adoption in
scheduler and placement studies are likely to inform future extensions of the
system and its measurement capabilities.

\section{Final Remarks}
\label{sec:conclusion-final-remarks}

Tycho is not intended as a universal solution for operational energy monitoring
in production environments. Its design deliberately prioritizes semantic clarity,
temporal fidelity, and explicit treatment of uncertainty over simplicity and
minimal deployment requirements. As a result, Tycho is best understood as a
research instrument rather than a turnkey observability tool.

Within this intended scope, the work presented in this thesis demonstrates that
accuracy-first energy measurement is both feasible and informative, even under
the constraints of heterogeneous hardware and imperfect telemetry. By enforcing
explicit semantics, closed accounting, and transparent handling of residual
uncertainty, Tycho enables a form of workload-level energy analysis that supports
reasoned interpretation rather than implicit inference. This shift from
estimation-oriented reporting toward semantically grounded measurement changes
not only what can be observed, but also how energy behavior in complex systems
can be understood.

More broadly, this thesis argues that energy should be treated as a first-class
system property, subject to the same rigor in measurement, interpretation, and
validation as performance or resource utilization. As energy efficiency becomes
an increasingly important concern in distributed systems research, approaches
that favor explicitness and interpretability over convenience will be essential
for producing reliable and actionable insights. Tycho contributes one such
approach, providing a foundation for future research that seeks to reason about
energy consumption with clarity, restraint, and scientific discipline.
