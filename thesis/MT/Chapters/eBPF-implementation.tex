\section{eBPF-Based CPU Time Attribution}
\label{sec:ebpf-architecture}

\subsection{Scope and Motivation}
\label{subsec:ebpf-scope-motivation}

The kernel-level \code{eBPF} subsystem in Tycho provides the foundation for process-level energy attribution.  
It captures CPU scheduling, interrupt, and performance-counter events directly inside the Linux kernel, translating them into continuous measurements of CPU ownership and activity.  
All higher-level aggregation and modeling occur in userspace; this section therefore focuses exclusively on the in-kernel instrumentation and the data it exposes.

Kepler’s original \code{eBPF} design offered a coarse but functional basis for collecting CPU time and basic performance metrics.  
Its \code{sched\_switch} tracepoint recorded process runtime, while hardware performance counters supplied instruction and cache data.  
However, the sampling cadence and aggregation logic were controlled from userspace, producing irregular collection intervals and temporal misalignment with energy readings.  
Kepler also treated all CPU time as a single undifferentiated category, omitting explicit representation of idle periods, interrupt handling, and kernel threads.  
As a result, a portion of the processor’s activity (often significant under I/O-heavy workloads) remained unaccounted for in energy attribution.

Tycho addresses these limitations through a refined kernel-level design.  
New tracepoints capture hard and soft interrupts, while extended per-CPU state tracking distinguishes between user processes, kernel threads, and idle execution.  
Each CPU maintains resettable bins that accumulate idle and interrupt durations within well-defined time windows, providing temporally bounded activity summaries aligned with energy sampling intervals.  
Cgroup identifiers are refreshed at every scheduling event to maintain accurate container attribution, even when processes migrate between control groups.  
The result is a stable, low-overhead data source that describes CPU usage continuously and with sufficient granularity to support fine-grained energy partitioning in the subsequent analysis.

\subsection{Baseline and Architecture Overview}
\label{subsec:ebpf-overview}

Kepler’s kernel instrumentation consisted of a compact set of \code{eBPF} programs that sampled process-level CPU activity and a few hardware performance metrics.  
The core tracepoint, \code{tp\_btf/sched\_switch}, captured context switches and estimated per-process runtime by measuring the on-CPU duration between successive events.  
Complementary probes monitored page cache access and writeback operations, providing coarse indicators of I/O intensity.  
Hardware performance counters (CPU cycles, instructions, and cache misses) were collected through \code{perf\_event\_array} readers, enabling approximate performance characterization at the task level.

While effective for general profiling, this setup lacked the temporal resolution and system coverage required for precise energy correlation.  
The sampling process was driven entirely from userspace, leading to irregular collection intervals, and idle or interrupt time was never observed directly.  
Consequently, CPU utilization appeared complete only from a process perspective, leaving kernel and idle phases invisible to the measurement pipeline.

Tycho extends this architecture into a continuous kernel-side monitoring system.  
Each CPU maintains an independent state structure recording its current task, timestamp, and execution context.  
This allows uninterrupted accounting of CPU ownership, even between user-space scheduling events.  
New tracepoints for hard and soft interrupts measure service durations directly in the kernel, ensuring that all processor activity (user, kernel, or idle) is captured.  
Dedicated per-CPU bins accumulate these times within fixed analysis windows, which the userspace collector periodically reads and resets.  
Process-level metrics are stored in an LRU hash map, while hardware performance counters remain integrated via existing PMU readers.

Data flows linearly from tracepoints to per-CPU maps and onward to the userspace collector, forming a continuous and low-overhead measurement path.  
This architecture transforms Kepler’s periodic snapshot model into a streaming telemetry layer that maintains temporal consistency and provides the necessary basis for accurate, time-aligned energy attribution.

\subsection{Kernel Programs and Data Flow}
\label{subsec:ebpf-programs}

Tycho’s \code{eBPF} subsystem consists of a small set of tracepoints and helper maps that together maintain a continuous record of CPU activity.  
Each program updates per-CPU or per-task data structures in response to kernel events, ensuring that all processor time is accounted for across user, kernel, and idle contexts.

\paragraph{Scheduler Switch}
The central tracepoint, \code{tp\_btf/sched\_switch}, triggers whenever the scheduler replaces one task with another.  
It computes the elapsed on-CPU time of the outgoing process and updates its entry in the \code{processes} map, which stores runtime, hardware counter deltas, and classification metadata such as \code{cgroup\_id}, \code{is\_kthread}, and command name.  
Hardware counters for instructions, cycles, and cache misses are read from preconfigured PMU readers at this moment, keeping utilization metrics temporally aligned with task execution.  
Each CPU also maintains a lightweight \code{cpu\_state} structure that records the last timestamp, currently active PID, and task type.  
When the idle task (PID 0) is scheduled, this structure accumulates idle time locally, allowing continuous accounting even between user-space sampling intervals.

\paragraph{Interrupt Handlers}
To capture system activity outside user processes, Tycho introduces tracepoints for hard and soft interrupts.  
Pairs of entry and exit hooks (\code{irq\_handler\_\{entry,exit\}} and \code{softirq\_\{entry,exit\}}) measure the time spent in each category by recording timestamps in the per-CPU state and adding the resulting deltas to dedicated counters.  
These durations are aggregated in \code{cpu\_bins}, a resettable per-CPU array that also stores idle time.  
At each collection cycle, userspace reads these bins, derives total CPU activity for the window, and resets them to zero for the next interval.

\paragraph{Page-Cache Probes}
Kepler’s original page-cache hooks (\code{fexit/mark\_page\_accessed} and \code{tp/writeback\_dirty\_folio}) are preserved.  
They increment per-process counters for cache hits and writeback operations, serving as indicators of I/O intensity rather than direct power consumption.

\paragraph{Supporting Maps and Flow}
All high-frequency updates occur in per-CPU or LRU hash maps to avoid contention.  
\code{pid\_time\_map} tracks start timestamps for active threads, enabling precise runtime computation during context switches.  
\code{processes} holds per-task aggregates, while \code{cpu\_states} and \code{cpu\_bins} manage temporal accounting per core.  
PMU event readers for cycles, instructions, and cache misses remain shared with Kepler’s implementation.  
At runtime, data flows from tracepoints to these maps and then to the userspace collector through batched lookups, forming a deterministic, lock-free telemetry path from kernel to analysis.

\subsection{Collected Metrics}
\label{subsec:ebpf-metrics}

The kernel \code{eBPF} subsystem exports a defined set of metrics describing CPU usage at process and system levels.  
These values are aggregated in kernel maps and periodically retrieved by the userspace collector for time-aligned energy analysis.  
Table~\ref{tab:ebpf-metrics} summarizes all metrics grouped by category.

\begin{table}[h]
\centering
\caption{Metrics collected by the kernel \code{eBPF} subsystem.}
\label{tab:ebpf-metrics}
\begin{tabular}{p{3cm} p{3.2cm} p{2.2cm} p{6.4cm}}
\toprule
\textbf{Metric} & \textbf{Source hook} & \textbf{Granularity} & \textbf{Description} \\
\midrule
\multicolumn{4}{l}{\textit{Time-based metrics}} \\[2pt]
Process runtime & \code{tp\_btf/sched\_switch} & per process & Elapsed on-CPU time accumulated at context switches. \\
Idle time & derived from \code{sched\_switch} & per CPU & Time with no runnable task (PID 0). \\
IRQ time & \code{irq\_handler\_{\{entry,exit\}}} & per CPU & Duration spent in hardware interrupt handlers. \\
SoftIRQ time & \code{softirq\_{\{entry,exit\}}} & per CPU & Duration spent in deferred kernel work. \\[4pt]

\multicolumn{4}{l}{\textit{Hardware-based metrics}} \\[2pt]
CPU cycles & PMU (\code{perf\_event\_array}) & per process & Retired CPU cycle count during task execution. \\
Instructions & PMU (\code{perf\_event\_array}) & per process & Retired instruction count. \\
Cache misses & PMU (\code{perf\_event\_array}) & per process & Last-level cache miss count; indicates memory intensity. \\[4pt]

\multicolumn{4}{l}{\textit{Classification and enrichment metrics}} \\[2pt]
Cgroup ID & \code{sched\_switch} & per process & Control group identifier for container attribution. \\
Kernel thread flag & \code{sched\_switch} & per process & Marks kernel threads executing in system context. \\
Page cache hits & \code{mark\_page\_accessed} & per process & Read or write access to cached pages; proxy for I/O activity. \\
IRQ vectors & \code{softirq\_entry} & per CPU & Frequency of specific soft interrupt vectors. \\
\bottomrule
\end{tabular}
\end{table}

Together these metrics form a coherent description of CPU activity.  
Time-based data quantify ownership of processing resources, hardware counters capture execution intensity, and classification attributes link activity to its origin.  
This dataset serves as the kernel-level foundation for energy attribution and higher-level modeling in userspace.
\subsection{Integration with Energy Measurements}
\label{subsec:ebpf-integration-energy}

The data exported from the kernel define how CPU resources are distributed among processes, kernel threads, interrupts, and idle periods during each observation window.  
When combined with energy readings obtained over the same interval, these temporal shares provide the basis for proportional energy partitioning.  
Instead of relying on statistical inference or coarse utilization averages, Tycho attributes energy according to directly measured CPU ownership.

Each process contributes its accumulated runtime and performance-counter deltas, while system activity and idle phases are derived from the per-CPU bins.  
The sum of these components represents the total active time observed by the processor, matching the energy sample boundaries defined by the timing engine.  
This alignment ensures that every joule of measured energy can be traced to a specific class of activity (user workload, kernel service, or idle baseline).  
Through this mechanism, the \code{eBPF} subsystem provides the precise temporal structure required for fine-grained, container-level energy attribution in the subsequent analysis stages.

\subsection{Efficiency and Robustness}
\label{subsec:ebpf-efficiency}

The kernel instrumentation is designed to operate continuously with negligible system impact while ensuring correctness across kernel versions.  
All high-frequency data reside in per-CPU maps, eliminating cross-core contention and locking.  
Each processor updates only its local entries in \code{cpu\_states} and \code{cpu\_bins}, while per-task data are stored in a bounded LRU hash that automatically removes inactive entries.  
Arithmetic within tracepoints is deliberately minimal (timestamp subtraction and counter increments only) so that the added latency per event remains near the measurement noise floor.

Userspace retrieval employs batched \code{BatchLookupAndDelete} operations, reducing system-call overhead and maintaining constant latency regardless of map size.  
Hardware counters are accessed through pre-opened \code{perf\_event\_array} readers managed by the kernel, avoiding repeated setup costs.  
This architecture allows the subsystem to record thousands of context switches per second while keeping CPU overhead low.

Correctness is maintained through several safeguards.  
CO-RE (Compile Once, Run Everywhere) field resolution protects the program from kernel-version differences in \code{task\_struct} layouts.  
Cgroup identifiers are refreshed only for the newly scheduled task, ensuring accurate container labeling even when group membership changes.  
The idle task (PID 0) and kernel threads are handled explicitly to prevent user-space misattribution, and the resettable bin design enforces strict temporal separation between sampling windows.  
Together, these measures yield a stable and version-tolerant tracing layer that can run indefinitely without producing inconsistent or overlapping samples.

\subsection{Limitations and Future Work}
\label{subsec:ebpf-limitations}

Although the extended \code{eBPF} subsystem provides comprehensive temporal coverage of CPU activity, several limitations remain.  
Its precision is ultimately bounded by the granularity of available energy telemetry, as energy readings must be averaged over fixed sampling windows to remain stable.  
Within shorter intervals, power fluctuations introduce noise that limits the accuracy of direct attribution.  

The current implementation also omits processor C-state and frequency information.  
While idle and active time are distinguished, variations in power state and dynamic frequency scaling are not yet represented in the collected data.  
Including tracepoints such as \code{power:cpu\_idle} and \code{power:cpu\_frequency} would enable finer correlation between CPU state transitions and power usage.  
Additionally, very short-lived processes may be evicted from the LRU map before collection, slightly undercounting transient workloads.