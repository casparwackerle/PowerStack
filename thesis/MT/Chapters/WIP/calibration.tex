\subsection{Delay Calibration}
Delay calibration determines the time interval between the start of a workload and the first measurable reaction in the corresponding hardware metric. Because Tycho itself does not execute workloads on the host and does not have direct access to specialised hardware resources, delay calibration must be performed by external scripts that run on the bare-metal node. Each calibration attempt begins with an idle period until the metric reaches a stationary state, followed by a controlled workload with a known start time. The delay is the earliest sample that exceeds the idle baseline by a detectable margin. Since many hardware metrics apply internal averaging or have irregular publish cycles, a single run is not sufficient. Multiple runs are required to obtain a stable distribution. Tycho uses either the minimum or the fifth percentile of observed delays. The minimum reflects the earliest possible reaction. The fifth percentile can be used when the minimum appears to be an outlier.

\subsubsection{GPU delay calibration}
GPU delay calibration uses a dedicated script that generates a controlled GPU workload and monitors NVML power readings. A preliminary attempt relied on gpu-burn\parencite{wilicc_gpu‐burn}, but this tool carries a non-negligible startup delay that obscures the true hardware reaction time. To address this, the calibration mechanism was reimplemented with Numba\parencite{numba_pydata_org}, which allows the script to launch a custom floating-point kernel with exact control over the workload timing.

The script alternates between idle and active periods. During idle periods, NVML power is sampled until the readings reach a stable baseline, and only the final portion of the idle window is used for statistical analysis. During active periods, the Numba kernel saturates the GPU's compute units while the script continually samples NVML power. The delay is identified as the first sample that exceeds the idle baseline by a small, adaptively computed threshold. Because NVML power reports are averaged over approximately one second, many runs are required to gather a suitable distribution of delays. The default configuration uses 15 seconds of idle time, 15 seconds of active workload, a sampling interval of 50 milliseconds, and 100 runs.

\subsubsection{RAPL}
No dedicated delay calibration is planned for RAPL. RAPL energy counters are updated internally at close-to-millisecond scale and are accessed through local kernel interfaces, so access latency is negligible. Although very early work criticised timing characteristics at sub-millisecond resolution\parencite{khan2018rapl}, later studies generally consider RAPL accurate for the time scales relevant to energy modelling. Tycho enforces a minimum collection interval of 50 milliseconds because RAPL readings are noisy at very short intervals\parencite{schone2024energy}. At this granularity any residual delay is small compared to the sampling window and does not meaningfully affect alignment. Explicit delay calibration would therefore provide minimal benefit.

\subsubsection{Redfish}
Redfish presents a fundamentally different problem. Power readings are published slowly, with irregular inter-arrival times, and may skip updates entirely. An earlier empirical study reported delays of roughly 200 milliseconds\parencite{wang2019empirical}, but also noted substantial variability across systems. Additional indeterminism arises from the network path and the unknown internal behaviour of the BMC. Since Tycho already mitigates staleness through its freshness mechanism, explicit delay calibration is neither feasible nor useful. Redfish is therefore treated as a coarse, low-resolution metric, appropriate for slow global trends but not for fine-grained timing.

\subsubsection{eBPF Metrics}
eBPF-based utilisation metrics behave differently. They are collected directly in kernel context and do not involve additional publish intervals or device-side buffering. Their effective delay is negligible relative to Tycho's sampling windows, so no delay calibration is required.
