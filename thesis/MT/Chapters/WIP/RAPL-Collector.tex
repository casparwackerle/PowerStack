\section{RAPL Collector Integration}
\label{sec:rapl_collector}

\subsection{Scope and Baseline}
\label{subsec:rapl_scope}

RAPL-based telemetry provides CPU energy readings at the level of logical power domains such as package, cores, uncore, and DRAM.  
Tycho reuses Kepler's existing RAPL implementation almost unchanged: discovery, scaling, and raw counter reads are still provided by the shared \code{components} package (via the \code{power}-interface), which wraps the kernel's \code{powercap} interface and related mechanisms.  
This collector therefore does not introduce a new hardware access path but instead focuses on integrating the existing RAPL metrics into Tycho's timing and buffering model.

\subsection{Architecture and Data Flow}
\label{subsec:rapl_architecture}

The collector is deliberately simple and stateless.  
On each engine tick, the \code{Collect} function:

\begin{enumerate}
  \item Verifies that system-wide RAPL collection is supported via\newline \code{components.IsSystemCollectionSupported()}.
  \item Obtains a snapshot of cumulative energy counters for all sockets and domains through \code{components.GetAbsEnergyFromNodeComponents()}.
  \item Translates the per-socket results into a \code{RaplTick} structure containing:
  \begin{itemize}
    \item a monotonic timestamp derived from \code{clock.Mono}, and
    \item a map from socket identifier to domain counters.
  \end{itemize}
  \item Pushes exactly one immutable \code{RaplTick} into the shared ring buffer.
\end{enumerate}

The tick stores raw, monotonically increasing counters in millijoules rather than precomputed deltas.  
Downstream analysis components compute inter-tick differences, perform wraparound handling, and align RAPL energy with CPU activity and platform-level power traces.  
This separation keeps the collector minimal while allowing the analysis layer to apply a consistent attribution strategy across all energy domains.

\subsection{Collected Metrics}
\label{subsec:rapl_metrics}

Table~\ref{tab:rapl-metrics} summarises the metrics stored in each \code{RaplTick}.\\  
All counters are subject to platform availability.

\begin{table}[H]
\centering
\small
\begin{tabular}{p{3.5cm} p{0.9cm} p{8.6cm}}
\toprule
\textbf{Metric} & \textbf{Unit} & \textbf{Description} \\
\midrule
\multicolumn{3}{l}{\textit{Per-socket energy counters}} \\[4pt]
\code{Pkg}    & mJ & Cumulative package energy per socket\newline(RAPL \code{PKG} domain). \\
\code{Core}   & mJ & Cumulative core energy per socket\newline (RAPL \code{PP0} domain), when available. \\
\code{Uncore} & mJ & Cumulative uncore energy per socket\newline (RAPL \code{PP1} or uncore domain), when available. \\
\code{DRAM}   & mJ & Cumulative DRAM energy per socket\newline (RAPL \code{DRAM} domain), if the platform exposes it. \\[6pt]

\multicolumn{3}{l}{\textit{Metadata}} \\[4pt]
\code{Source}    & -- & Identifier of the active RAPL backend (for example \code{powercap})\\
\code{Sockets}   & -- & Map from socket identifier to the corresponding set of domain counters. \\
\code{SampleMeta.Mono} & -- & Monotonic timestamp assigned by Tycho's timing engine at the moment of collection. \\
\bottomrule
\end{tabular}
\caption{Metrics exported by the RAPL collector per \code{RaplTick}.}
\label{tab:rapl-metrics}
\end{table}

These metrics provide a compact, hardware-backed view of CPU and memory-subsystem energy at socket granularity.  
Combined with the timing guarantees of Tycho's engine and ring buffer, they form the CPU-related energy baseline against which per-process activity and platform-level power are later correlated.

\subsection{Limitations and Reuse}
\label{subsec:rapl_limitations}

Because RAPL domain availability is hardware dependent, not all fields in
Table~\ref{tab:rapl-metrics} are present on every platform.  
Some systems expose only package-level counters, while others additionally
provide core, uncore, or DRAM domains.  
The collector reports only what the hardware exposes and does not attempt to
approximate or reconstruct missing values.

Beyond this domain variability, no further limitations apply.  
The collector is intentionally minimal and serves as a timing-aware wrapper
around the raw counters, providing reliable cumulative energy readings without
adding intermediate modelling or transformation.
