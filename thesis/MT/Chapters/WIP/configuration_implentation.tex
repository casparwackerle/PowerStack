\section{Configuration Management}
\label{subsec:tycho_config}

\subsection{Overview and Role in the Architecture}
\label{subsubsec:tycho_config_overview}

Tycho adopts a simple, centralized configuration layer that is initialized during exporter startup and made globally accessible through typed structures.
This layer defines all runtime parameters controlling timing, collection, and analysis behaviour.
It serves as the interface between user-defined settings and the internal scheduling and buffering logic described in \S~\ref{sec:tycho_timing_engine}.

The configuration is loaded once at startup, combining defaults, environment variables, and optional overrides passed through Helm or local flags.
Its purpose is not to support dynamic reconfiguration, but to provide deterministic, reproducible operation across (experimental) runs.
No backward compatibility with previous Kepler versions is maintained.

\subsection{Configuration Sources}
\label{subsubsec:tycho_config_sources}

Configuration values can be provided in three ways:
first, through a \code{values.yml} file during Helm installation,
second, as command-line flags for local or debugging builds,
and third, via predefined environment variables that act as defaults.

During startup, Tycho sequentially evaluates these sources in fixed order—
defaults are loaded first, then environment variables,
followed by any user-supplied overrides.
The resulting configuration is stored in memory and printed once for verification.
After initialization, all components reference the same in-memory configuration,
ensuring consistent behaviour across collectors and analysis modules.

\subsection{Implementation and Environment Variables}
\label{subsubsec:tycho_config_env}
The configuration implementation in Tycho closely follows the approach used in Kepler~v0.9.0.
Each configuration key is mapped to an environment variable, which is resolved at startup through dedicated lookup functions.
If no variable is set, the corresponding default value is applied.
This mechanism enables flexible configuration without external dependencies or complex parsing logic.
All variables are read once during initialization, after which they are cached in typed configuration structures.
This guarantees consistent operation even if environment variables change later, since Tycho is not designed for live reconfiguration.
The configuration layer is invoked before the collectors and timing engine are instantiated,
ensuring that parameters such as polling intervals, buffer sizes, or analysis triggers are available to all components from the first cycle onward.

\subsubsection{Validation and Normalization at Startup}
\label{subsubsec:tycho_config_validate}

During initialization, Tycho validates all user inputs and normalizes them to a consistent, safe configuration.
First, basic bounds are enforced: the global timebase quantum must be positive, non-negative values are required for all periods and delays, and missing essentials fall back to minimal defaults.
Trigger coherence is then checked. If \code{redfish} is selected while the Redfish collector is disabled, Tycho switches to the timer trigger and ensures a valid interval. Unknown triggers default to \code{timer}.

All periods and delays are aligned to the global quantum so that scheduling, buffering, and analysis operate on a common time grid.
The analysis wait \code{DelayAfterMs} is raised if needed to cover the longest enabled per-source delay.
Buffer sizing is derived from the slowest effective acquisition path (poll period plus delay) and the analysis wait, with a small safety margin.
If Redfish is enabled, its heartbeat requirement is included to guarantee coverage.
Sanity checks also ensure plausible Redfish cadence and warn if no collectors are enabled.
Non-fatal environment hints (for example the RAPL powercap path) are reported at low verbosity.

The result is a single, internally consistent configuration snapshot.
Adjustments are announced once at startup to aid reproducibility while avoiding log noise.

\subsection{Evolution in Newer Kepler Versions}
\label{subsubsec:tycho_config_evolution}

Subsequent Kepler releases (v0.10.0 and later) have replaced the environment-variable system with a unified configuration interface based on CLI flags and YAML files.
This modernized approach simplifies configuration management and aligns better with Kubernetes conventions, providing clearer defaults and validation at startup.

Tycho intentionally retains the v0.9.0 model to maintain structural continuity with its experimental foundation.
Since configuration handling is not a research focus, adopting the newer scheme would add complexity without scientific benefit.
Nevertheless, the newer Kepler design confirms that Tycho’s configuration logic can be migrated with minimal effort if long-term maintainability becomes a requirement.

\subsection{Available Parameters}
\label{subsubsec:tycho_config_parameters}

All parameters are read at startup and remain constant throughout execution.
The following table~\ref{tab:tycho_config_parameters} summarizes the user-facing configuration variables with their default values and functional scope.
Internal or experimental parameters are omitted for clarity.


XXXXXXXXXXXXXXXXXXXXXXXXXXXXXXXXXXXXXXXXXXXXXXXXXXXXXXXXXXXXXXXXXXXXXXXXXXXXXXXXXXXXXXXXXXXXXXXXXXXXXXXXXXXXXXXXXXXXXXXXXXXXXXXXXXXXXXXXXXXXXXXXXXXXXX
CHECK THIS TABLE BEFORE HANDIN, THIS NEEDS CLEANUP\\
XXXXXXXXXXXXXXXXXXXXXXXXXXXXXXXXXXXXXXXXXXXXXXXXXXXXXXXXXXXXXXXXXXXXXXXXXXXXXXXXXXXXXXXXXXXXXXXXXXXXXXXXXXXXXXXXXXXXXXXXXXXXXXXXXXXXXXXXXXXXXXXXXXXXXX
\begin{table}[h]
\centering
\tiny
\begin{tabular}{p{4.8cm} p{1.2cm} p{7cm}}
\toprule
\textbf{Variable} & \textbf{Default} & \textbf{Description} \\
\midrule
\multicolumn{3}{l}{\textit{Collector enable flags}} \\[4pt]
\code{TYCHO\_COLLECTOR\_ENABLE\_BPF} & \code{true} & Enables eBPF-based process metric collection. \\
\code{TYCHO\_COLLECTOR\_ENABLE\_RAPL} & \code{true} & Enables RAPL energy counter collection. \\
\code{TYCHO\_COLLECTOR\_ENABLE\_GPU} & \code{true} & Enables GPU power telemetry collection. \\
\code{TYCHO\_COLLECTOR\_ENABLE\_REDFISH} & \code{true} & Enables Redfish-based BMC power collection. \\[4pt]

\multicolumn{3}{l}{\textit{Timing and delays}} \\[4pt]
\code{TYCHO\_TIMEBASE\_QUANTUM\_MS} & \code{1} & Base system quantum (ms) defining the global monotonic time grid. \\
\code{TYCHO\_RAPL\_POLL\_MS} & \code{50} & RAPL polling interval (ms). \\
\code{TYCHO\_GPU\_POLL\_MS} & \code{200} & GPU telemetry polling interval (ms). \\
\code{TYCHO\_REDFISH\_POLL\_MS} & \code{1000} & Redfish polling interval (ms); should be below BMC publish cadence. \\
\code{TYCHO\_RAPL\_DELAY\_MS} & \code{0} & Expected delay between workload change and RAPL visibility (ms). \\
\code{TYCHO\_GPU\_DELAY\_MS} & \code{200} & Expected delay between workload change and GPU visibility (ms). \\
\code{TYCHO\_REDFISH\_DELAY\_MS} & \code{0} & Expected delay between workload change and Redfish visibility (ms). \\
\code{TYCHO\_REDFISH\_HEARTBEAT\_MAX\_GAP\_MS} & \code{3000} & Maximum tolerated gap between consecutive Redfish samples (ms). \\[4pt]

\multicolumn{3}{l}{\textit{Autotuning controls}} \\[4pt]
\code{TYCHO\_RAPL\_POLL\_AUTOTUNE} & \code{true} & Enables automatic calibration of RAPL polling interval. \\
\code{TYCHO\_RAPL\_DELAY\_AUTOTUNE} & \code{true} & Enables automatic calibration of RAPL delay. \\
\code{TYCHO\_GPU\_POLL\_AUTOTUNE} & \code{true} & Enables automatic calibration of GPU polling interval. \\
\code{TYCHO\_GPU\_DELAY\_AUTOTUNE} & \code{true} & Enables automatic calibration of GPU delay. \\
\code{TYCHO\_REDFISH\_POLL\_AUTOTUNE} & \code{true} & Enables automatic calibration of Redfish polling interval. \\
\code{TYCHO\_REDFISH\_DELAY\_AUTOTUNE} & \code{true} & Enables automatic calibration of Redfish delay. \\[4pt]

\multicolumn{3}{l}{\textit{Analysis parameters}} \\[4pt]
\code{TYCHO\_ANALYSIS\_TRIGGER} & \code{"timer"} & Defines analysis trigger: \code{redfish} or \code{timer}. \\
\code{TYCHO\_ANALYSIS\_EVERY\_SEC} & \code{15} & Interval for timer-based analysis (s). \\
\code{TYCHO\_ANALYSIS\_DETECT\_LONGEST\_DELAY} & \code{false} & Enables detection of the longest observed metric delay. \\
\bottomrule
\end{tabular}
\caption{User-facing configuration variables available in Tycho.}
\label{tab:tycho_config_parameters}
\end{table}

In summary, Tycho’s configuration layer provides a single, immutable snapshot of
all timing, polling, and analysis parameters.  
It is loaded once at startup, validated, normalized to a shared timebase, and
propagated to all collectors and analysis modules.  
The design prioritizes reproducibility and correctness: no live reconfiguration
is supported, and all behaviour is derived from explicit user input or well-defined
defaults.  
This central configuration snapshot ensures that the timing engine, collectors,
and buffering subsystem operate in a coherent, predictable manner throughout the
entire runtime.
