\section{Redfish Collector Integration}
\label{sec:redfish_collector}

The Redfish collector retrieves node-level power measurements from the server's
Baseboard Management Controller (BMC) through the Redfish API.  As an
out-of-band source, it complements in-band telemetry such as RAPL by providing a
hardware-validated view of total chassis power.  Tycho integrates these
measurements into its global timing framework, ensuring that BMC-sourced data
can be correlated with all other collectors using a common monotonic clock.

\subsection{Overview and Objectives}
\label{subsec:redfish_overview}

Redfish power readings are updated asynchronously and often with vendor-specific
timing behaviour.  
Tycho therefore focuses on robustness, controlled polling, and precise
timestamping rather than high-frequency sampling.  The collector executes a
single BMC query at the global engine cadence, extracts the current chassis
power value, and emits a sample only when necessary.  All readings are
timestamped with Tycho's monotonic timebase and annotated with freshness
metadata to express their temporal accuracy.

\subsection{Baseline in Kepler}
\label{subsec:redfish_baseline}

Kepler's original Redfish integration consisted of a periodic background ticker
that queried the BMC at fixed intervals, typically every sixty seconds.  It
retrieved only instantaneous chassis power and made no distinction between new
and repeated measurements.  No mechanism existed to detect stale data, associate
readings with BMC timestamps, or align Redfish values with other collectors.
The Redfish implementation was therefore sufficient for coarse system-power
reporting, but not suitable for the finer temporal structure required by Tycho.

\subsection{Refactoring and Tycho Extensions}
\label{subsec:redfish_refactor}

\subsubsection{Timing Ownership and Polling Control}
\label{subsubsec:redfish_timing}

Tycho removes Kepler's internal Redfish ticker entirely.  
Polling is performed exclusively by Tycho's global timing engine, which invokes
the collector once per engine tick.  This ensures that all Redfish interactions
occur in lockstep with the rest of the measurement pipeline and that every
sample can be aligned unambiguously with simultaneous energy and utilization
values from other collectors.

The polling frequency itself is never altered by the Redfish collector.
Tycho maintains deterministic, externally selected polling intervals regardless
of BMC behaviour.

\subsubsection{Sequence-Based Newness Detection}
\label{subsubsec:redfish_newness}

Many BMCs return identical payloads for extended periods and only publish new
power values intermittently.  To avoid emitting redundant samples, Tycho uses a
per-chassis sequence number provided by the Redfish client.  Each measurement is considered fresh only when this sequence number differs from the one observed in the previous engine tick.

No header-based (for example \code{ETag}) or value-delta heuristics are used.
Newness is driven solely by this explicit sequence value, ensuring a consistent
and vendor-agnostic update mechanism.

\subsubsection{Heartbeat Mechanism and Freshness Metric}
\label{subsubsec:redfish_heartbeat}

Because BMC refresh cycles can be irregular, Tycho maintains a heartbeat window
for each chassis.  If no new Redfish sample has appeared within this window, the
collector emits a heartbeat sample that carries forward the last known power
value.  This prevents gaps in the power time series while avoiding unnecessary
fabrication of data.

Each emitted sample includes a freshness metric, defined as the time difference
between the BMC-reported timestamp (when available) and the local collection time.  This value quantifies the latency of the underlying Redfish pipeline.  
Freshness does not influence newness detection; it is an informational quality
indicator for the analysis layer.

\subsubsection{Fixed vs Auto Heartbeat Mode}
\label{subsubsec:redfish_autopoll}

Tycho supports two Redfish operating modes, selected by
\code{TYCHO\_REDFISH\_POLL\_AUTOTUNE}.  Both modes perform BMC polling strictly
at the global engine cadence.

In fixed mode, the heartbeat window is defined entirely by the user through
\code{TYCHO\_REDFISH\_HEARTBEAT\_MS}.  
This guarantees deterministic behaviour and is appropriate for reproducible
benchmarks.

In auto mode, Tycho adapts the heartbeat window to the BMC's observed update
pattern.  
Whenever a fresh sequence number is detected, the collector measures the
inter-arrival duration to the previous update and stores it in a sliding window.
The median of this window is taken as the representative publication period, and
the heartbeat timeout is set to approximately \(1.5\) times this value, clamped
between conservative bounds.  
This technique aligns Tycho's emission behaviour with the hardware's natural
update rhythm while retaining deterministic polling.

\subsection{Collected Metrics}
\label{subsec:redfish_metrics}

The Redfish collector emits one record per chassis whenever a new measurement is detected or a heartbeat event occurs.  Each record is timestamped using Tycho's monotonic clock and annotated with metadata relevant for temporal correlation.
Table~\ref{tab:redfish-metrics} summarises the collected fields.

\begin{table}[h]
\centering
\begin{tabular}{p{3cm} p{1cm} p{9cm}}
\toprule
\textbf{Metric} & \textbf{Unit} & \textbf{Description} \\
\midrule
\multicolumn{3}{l}{\textit{Primary power metric}} \\[4pt]
\code{PowerWatts} & W & Instantaneous chassis power reported by the BMC. \\[4pt]

\multicolumn{3}{l}{\textit{Temporal and identity metadata}} \\[4pt]
\code{ChassisID} & - & Identifier of the chassis or enclosure. \\
\code{Seq} & - & Server-provided sequence number indicating new measurements. \\
\code{SourceTime} & s & Timestamp provided by the BMC, if available. \\
\code{CollectorTime} & s & Local collection time of the measurement. \\
\code{FreshnessMs} & ms & Difference between \code{SourceTime} and \code{CollectorTime}. \\
\bottomrule
\end{tabular}
\caption{Metrics collected by the Redfish collector.}
\label{tab:redfish-metrics}
\end{table}

Energy values such as \code{EnergyMilliJ} are computed downstream by integrating the power series and are not directly produced by the collector.

\subsection{Integration and Data Flow}
\label{subsec:redfish_integration}

The Redfish collector is a passive component within Tycho's unified collection
pipeline.  At each engine tick it performs one BMC query, evaluates newness and
heartbeat conditions, attaches monotonic timestamps, and writes the resulting
record into a synchronized ring buffer.  
No additional processing occurs within the collector; all energy computations
and correlations across metrics are performed in Tycho's analysis stage.

\subsection{Accuracy and Robustness Improvements}
\label{subsec:redfish_accuracy}

Tycho improves Redfish robustness by combining explicit sequence-based newness
tracking, adaptive heartbeat windows, and freshness estimation.  
Sequence tracking ensures that repeated BMC responses do not produce redundant
samples.  
Adaptive heartbeat windows accommodate BMCs with slow or irregular update
cycles.  
Freshness metadata exposes the latency of Redfish timestamps, allowing the
analysis layer to assess the temporal reliability of each reading.

These measures provide stable and chronologically consistent power telemetry
across heterogeneous BMC implementations.

\subsection{Limitations}
\label{subsec:redfish_limitations}

Redfish sampling remains limited by the update rate and timestamp quality of the underlying BMC.  Most implementations refresh power data at intervals on the order of one to two seconds.  Redfish does not report component-level power
breakdowns; only total chassis power is available.  For fine-grained
attribution, Tycho relies on complementary in-band collectors such as RAPL and
eBPF.
