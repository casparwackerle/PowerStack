
\chapter{Introduction}
\label{ch:introduction}

XXXXXXXXXXXXXXXXXXXXXXXXXXXXXXXXXXXXXXXXXXXXXXXXXXXXXXXXXXXXXXXXXXXXXXXXXXXXXXXXXXXXXXXXXXXXXXXXXXXXXXXXXXXXXXXX
REVISE ENTIRE CHAPTER LATER
XXXXXXXXXXXXXXXXXXXXXXXXXXXXXXXXXXXXXXXXXXXXXXXXXXXXXXXXXXXXXXXXXXXXXXXXXXXXXXXXXXXXXXXXXXXXXXXXXXXXXXXXXXXXXXXX

\section{Motivation}
\label{sec:intro_motivation}

Energy consumption in data centers continues to rise as demand for compute-intensive and latency-sensitive services increases. Modern cloud platforms host diverse workloads such as machine learning inference, analytics pipelines, and high-density microservices, all of which collectively contribute to a growing global electricity footprint. Container orchestration frameworks amplify these trends by enabling dense consolidation of workloads across shared servers. While this improves resource efficiency, it also introduces abstraction layers that obscure the relationship between workload behaviour and physical energy use.

As interest in sustainable cloud operations intensifies, there is increasing demand for precise, workload-level energy visibility. Fine-grained and reproducible energy measurements are essential for research domains such as performance engineering, scheduling, autoscaling, and the design of energy-aware systems. Existing tools provide valuable approximations but prioritise portability and low operational overhead, and therefore do not target the upper bounds of measurement fidelity. Research environments, by contrast, require methodologies that prioritise accuracy, control, and verifiability over deployability.

This thesis is motivated by the need for an accuracy-focused measurement approach that supports rigorous experimental work on containerised systems. Rather than proposing new optimisation mechanisms, this work concentrates on establishing a reliable methodological foundation for observing and analysing workload-induced energy consumption in controlled settings.

\section{Problem Context}
\label{sec:intro_context}

Modern multi-tenant servers host many short-lived and highly dynamic workloads that execute concurrently and compete for shared hardware resources. On such systems, the aggregate power draw represents the combined activity of numerous interacting subsystems, while the contributions of individual workloads remain deeply entangled. Containerisation further complicates this picture: processes belong to containers, containers belong to pods, and pods may change state rapidly under orchestration. These abstractions improve system management but obscure how computational activity translates into power consumption.

At the same time, servers expose a heterogeneous collection of telemetry sources. Each source reflects different aspects of hardware behaviour, updates at its own cadence, and provides only a partial view of system activity. Because workload state changes and telemetry updates occur independently, they do not naturally align in time. The resulting temporal misalignment limits the reliability of workload-level energy attribution and leads to uncertainty in short-duration or phase-sensitive analyses.

Kubernetes introduces additional challenges. Workloads may start and terminate within milliseconds, metadata may appear with delays, and lifecycle events may interleave in complex ways. Existing tools often rely on coarse sampling windows or heuristic models that mask these inconsistencies. While sufficient for operational monitoring, such abstractions constrain the achievable accuracy in research settings. An accuracy-oriented approach requires explicit treatment of timing, metadata consistency, and correlation across heterogeneous measurement sources.

\section{Position Within Previous Research}
\label{sec:intro_prevwork}

This thesis builds upon two earlier stages of work. The implementation-focused VT1 project developed an initial measurement pipeline and explored practical aspects of collecting hardware and system-level metrics in a Kubernetes environment. The subsequent VT2 project examined the state of the art in server-level energy measurement, validated the behaviour of commonly used telemetry sources, and identified methodological and technical limitations in existing tools such as Kepler. Both works are included in the appendix as supporting material.

The present thesis integrates these earlier insights but does not repeat them. Instead, it synthesises the essential findings from VT2 in a condensed form (\chapref{ch:background}), and introduces the conceptual foundations required to reason about accurate energy attribution (\chapref{ch:concepts}). These chapters provide the background necessary to understand the accuracy-focused architecture developed later in this thesis.

\section{Problem Statement}
\label{sec:intro_problem_statement}

Accurately determining how much energy individual workloads consume in a Kubernetes cluster remains a challenging open problem. Clusters host many short-lived and overlapping workloads whose behaviour evolves rapidly, while server-level power telemetry is exposed through heterogeneous interfaces that update asynchronously and lack consistent timestamps. These timing mismatches, combined with the abstraction layers introduced by container orchestration, obscure the relationship between workload activity and physical energy use. Existing approaches provide high-level estimates but cannot deliver the temporal alignment, attribution fidelity, or reproducibility required for rigorous experimental analysis. This thesis therefore addresses the problem of designing a measurement methodology and prototype system capable of producing time-aligned, workload-level energy attribution with sufficient accuracy for research environments.

\section{Goals of This Thesis}
\label{sec:intro_goals}

The overarching goal of this thesis is to develop an accuracy-focused approach for measuring energy consumption in Kubernetes-based environments. To achieve this, the work pursues four concrete objectives:

\begin{itemize}
    \item \textbf{Methodological objective:} Define a measurement methodology that aligns heterogeneous telemetry sources with dynamic workload behaviour under a unified temporal model suitable for controlled research settings.
    
    \item \textbf{Architectural objective:} Design an accuracy-first system architecture that explicitly handles timing, metadata consistency, and correlation across diverse metrics without relying on heuristic abstractions.
    
    \item \textbf{Prototype objective:} Implement a research prototype that realises this architecture on commodity server hardware and integrates workload metadata, timing information, and server-wide telemetry into a coherent measurement pipeline.
    
    \item \textbf{Foundational objective for future work:} Establish the methodological and architectural basis for subsequent validation studies that will evaluate measurement fidelity and explore trade-offs between accuracy, overhead, and operational constraints.
\end{itemize}


\section{Research Questions}
\label{sec:intro_research_questions}

\begin{enumerate}
    \item How reliably can an accuracy-focused measurement approach capture and represent workload-induced variations in energy consumption within dynamic, multi-tenant Kubernetes environments?
    \item To what extent does a unified timing and attribution methodology improve the consistency and interpretability of workload-level energy measurements compared to existing estimation-oriented approaches?
    \item In which contexts does high-fidelity energy measurement provide meaningful benefits for research and experimental analysis, and what trade-offs arise between accuracy, overhead, and operational constraints?
\end{enumerate}

\section{Contributions}
\label{sec:intro_contributions}

This thesis makes several conceptual and methodological contributions to the study of energy measurement in container-orchestrated environments. First, it introduces an accuracy-focused measurement approach that prioritizes temporal consistency, reproducibility, and the faithful representation of workload behaviour. The work defines a methodology for unifying heterogeneous sources of server telemetry under a shared timing model, enabling coherent interpretation of workload activity and system-level energy use. 

A second contribution is the development of a prototype system that operationalizes this methodology and provides a concrete platform for exploring the limits of high-fidelity energy measurement in Kubernetes-based environments. The prototype integrates workload metadata, timing information, and server-wide telemetry into a coherent measurement pipeline designed for research and controlled experimentation.

Third, the thesis establishes a foundation for reliable workload-level attribution by describing a structured process for correlating dynamic workload behaviour with system energy consumption. This provides a basis for analysing short-lived workload phases, transient resource usage patterns, and other phenomena that require fine-grained temporal alignment.

Finally, the work prepares the methodological groundwork for subsequent validation studies by outlining experimental procedures, calibration strategies, and evaluation principles suited to accuracy-oriented measurement. Together, these contributions advance the methodological state of the art and offer a practical reference point for future research on energy transparency in modern cloud infrastructures.

\section{Scope and Boundaries}
\label{sec:intro_scope}

This thesis focuses on high-level principles and methods for energy measurement in multi-tenant server environments. The primary scope includes conceptual design, prototype development, and preparation of the methodological foundation for subsequent evaluation work. The emphasis is on accuracy, reproducibility, and consistency rather than operational deployability or production-grade integration.

Several areas remain outside the scope of this work. The thesis does not propose scheduling policies, predictive models, or system-level optimisation mechanisms. It does not modify Kubernetes or introduce changes to cloud operators' workflows. The prototype developed in this thesis is intended for controlled research environments and does not aim to provide a turnkey solution for general-purpose use. The work assumes access to a server environment where low-level telemetry and measurement interfaces are accessible under suitable conditions.

\section{Origin of the Name ``Tycho''}
\label{sec:intro_name}

The prototype developed in this thesis is named \textit{Tycho}, a reference to the astronomer Tycho Brahe. Brahe is known for producing exceptionally precise astronomical measurements, which later enabled Johannes Kepler to formulate the laws of planetary motion. The naming reflects a similar relationship: while the upstream \textit{Kepler} project focuses on modelling and estimation, this thesis explores the upper bounds of measurement accuracy. Tycho thus signals both continuity with prior work and a shift toward an accuracy-first design philosophy.

\section{Methodological Approach}
\label{sec:intro_methodology}

% PLACEHOLDER: to be filled with a high-level description of the methodological 
% approach. The focus will be on conceptual design, prototype implementation, and 
% controlled calibration and validation studies, without technical detail.

\section{Thesis Structure}
\label{sec:intro_structure}

% PLACEHOLDER: to be filled with a short roadmap describing the structure 
% of Chapters 2–8 in a concise, high-level manner.
