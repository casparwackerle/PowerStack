
\chapter{Introduction}
\label{ch:introduction}
% This chapter introduces the motivation, problem context, research questions,
% and contributions of this thesis. It also provides contextual information
% about previous work, audience, methodology, and other relevant perspectives.

%--------------------------------------------------------------------
\section{Motivation}
\label{sec:intro_motivation}

\section{Motivation}
\label{sec:intro_motivation}

Global data center energy consumption is increasing rapidly, driven by the rising demand for
compute-heavy workloads such as artificial intelligence and large-scale analytics.
Data centers consumed roughly 1.5\% of global electricity in 2024
(about 415~TWh) and are expected to more than double their consumption by 2030
\parencite{iea2025energyai}.  
At the same time, traditional sources of efficiency improvement, such as Moore's law and 
Dennard scaling, are slowing down \parencite{tomshardware2023mooreslaw, cartesian2013dennard}, 
while gains from infrastructure optimizations approach diminishing returns 
\parencite{uptime2023pue, masanet2020}.

Containerized workloads form an increasingly large share of this growing footprint.
They provide lightweight and scalable deployment mechanisms \parencite{Potdar2020}, and
Kubernetes has become the dominant platform for orchestrating such workloads at scale.
Despite their operational benefits and generally low overhead \parencite{Morabito2015},
containers make energy transparency difficult. Their shared-resource architecture and
multi-layer abstraction obscure the relationship between individual workloads and the
physical energy they consume, which complicates research on energy efficiency. As interest
in energy-aware computing grows, accurate and fine-grained measurement becomes essential
for research, performance engineering, and scheduling. Existing tools offer useful
approximations but rely heavily on heuristic models and asynchronous telemetry, which
limits their validity and reproducibility.

\textit{Kepler}\parencite{kepler_current}, a CNCF-backed project under active development, has become a widely adopted energy observability tool for Kubernetes clusters. Its design prioritizes portability,
low overhead, and safe deployment across diverse environments, which makes it highly
suitable for operational use at scale. A detailed analysis of its methodology and
design philosophy is provided in \hyperref[appendix_vt2]{Appendix~A}. In this context, Tycho is not intended
to compete with Kepler or replace it. Instead, Tycho explores a complementary space:
the upper bound of what measurement accuracy is achievable when higher-privilege
monitoring methods, synchronized server-wide telemetry, and more fine-grained event-time 
alignment are permitted. While such techniques are often impractical for general-purpose
deployments, they offer substantial value for research settings and large-scale R\&D
environments that require precise and reproducible energy measurements. Tycho is an
attempt to fill this accuracy-focused niche by implementing methods that prioritize
measurement fidelity over broad deployability.

%--------------------------------------------------------------------
\section{Challenges and Limitations of Existing Approaches}
\label{sec:intro_teaser_limitations}

Current approaches to workload-level energy estimation provide useful insights but remain
limited in accuracy and reproducibility. Tools such as Kepler and Scaphandre\parencite{scaphandre_github} correlate resource utilization metrics with node-level energy telemetry, yet they operate under
constraints that make fine-grained attribution difficult. Many systems depend on
asynchronous and low-frequency sampling, rely on heterogeneous telemetry sources with
differing clock domains, or employ heuristic models that introduce uncertainty. Purpose-built,
high-accuracy measurement hardware exists, but its cost and operational overhead prevent
widespread adoption in production environments. As a result, most available
solutions offer high-level trends rather than precise, time-aligned measurements required
for rigorous analysis. A more detailed discussion of these limitations is provided in
\S~\ref{ch:background_and_related_work} and further expanded in \hyperref[appendix_vt2]{Appendix~A}.


%--------------------------------------------------------------------
\section{Problem Context}
\label{sec:intro_context}
% Describe the environment in which this problem exists.
% Explain multi-tenant Kubernetes clusters and their dynamic nature.
% Discuss heterogeneity of telemetry sources (RAPL, Redfish, GPU, eBPF).
% Explain why asynchronous sampling and inconsistent timestamps make reliable attribution difficult.
% Introduce the complexity of mapping processes to containers in Kubernetes.

%--------------------------------------------------------------------
\section{Problem Statement}
\label{sec:intro_problem_statement}
% Provide a concise and precise definition of the primary research problem.
% State that existing tools cannot deliver server-wide, time-aligned, and verifiable
% energy attribution at process level.
% Position Tycho as an accuracy-first approach that addresses this gap.

%--------------------------------------------------------------------
\section{Goals of This Thesis}
\label{sec:intro_goals}
% Explain high-level goals:
% - Develop an accuracy-first architecture for energy measurement.
% - Unify heterogeneous telemetry sources under a common timing model.
% - Build a reliable container-level attribution pipeline.
% - Design a verifiable and transparent methodology.
% - Prepare a foundation for the validation experiments in the next thesis (VT3).

%--------------------------------------------------------------------
\section{Research Questions}
\label{sec:intro_research_questions}
% Add the central research questions, for example:
% - How can telemetry sources with different sampling behaviours be aligned to a shared time base?
% - What accuracy is achievable with commodity server hardware?
% - Which telemetry sources are essential for reliable attribution?
% - How can short-lived utilization bursts be captured?
% - How can attribution errors be detected and reduced?

%--------------------------------------------------------------------
\section{Contributions}
\label{sec:intro_contributions}
% Summarize the main contributions of this thesis:
% - Definition of an accuracy-first design philosophy for energy measurement.
% - Introduction of a unified timing engine with monotonic timestamps.
% - Implementation of independent collectors (RAPL, GPU, Redfish, eBPF).
% - Ring buffer based synchronization model.
% - Metadata correlation engine for Kubernetes containers.
% - Prototype implementation called Tycho.
% - Integration of VT1 and VT2 as appendices for methodology and background.

%--------------------------------------------------------------------
\section{Positioning Within Previous Work}
\label{sec:intro_positioning}
% Explain how this thesis relates to VT1 and VT2.
% - VT1: Implementation-focused feasibility study for energy measurement in Kubernetes.
% - VT2: Theoretical deep dive into state-of-the-art methods, challenges, and attribution models.
% - Current thesis: Consolidates theory and implementation, introduces Tycho,
%   and prepares the groundwork for accuracy-focused validation in VT3.
% Emphasize continuity while avoiding repetition, supported by the appendices.

%--------------------------------------------------------------------
\section{Scope and Boundaries}
\label{sec:intro_scope}
% Clarify what this thesis covers and what it does not:
% INCLUDED:
% - Server-wide energy measurement (CPU, GPU, platform).
% - Process-level attribution using Kubernetes metadata.
% - Development of a unified timing and synchronization model.
% - Design and implementation of Tycho as a prototype system.
%
% EXCLUDED:
% - Providing a production-ready autoscaler or scheduler.
% - Prediction models or long-term forecasting of energy consumption.
% - Modifying Kubernetes itself.
% - Vendor-specific proprietary APIs beyond RAPL, NVML, and Redfish.
% - Energy-aware orchestration policies.
%
% The goal is a research prototype focused on accuracy, not a turnkey product.

%--------------------------------------------------------------------
\section{Target Audience}
\label{sec:intro_audience}
% Explain who Tycho is intended for.
% Tycho's accuracy-first design comes with drawbacks:
% - It is more complex than typical observability tools.
% - It requires specific privileges (eBPF, RAPL access, or BMC polling).
% - It produces low-level telemetry rather than high-level abstractions.
%
% Therefore, Tycho is primarily aimed at:
% - Academic researchers studying energy efficiency.
% - Infrastructure and performance engineering teams in hyperscalers.
% - Experimental environments where measurement accuracy is more important than ease of deployment.
%
% For typical operational users or standard SRE environments, Tycho may be excessive.

%--------------------------------------------------------------------
\section{Origin of the Name ``Tycho''}
\label{sec:intro_name}
% Provide a short explanation of the naming:
% Kepler is the upstream project, named after Johannes Kepler.
% Tycho Brahe was Kepler's mentor and provided the precise astronomical measurements
% that allowed Kepler to derive his laws of planetary motion.
%
% The analogy:
% - Kepler (the tool) builds models and estimates.
% - Tycho (this project) focuses on precise and accurate raw measurement.
%
% The naming signals both continuity and a philosophical shift toward accuracy-first design.

%--------------------------------------------------------------------
\section{Methodological Approach}
\label{sec:intro_methodology}
% Provide a short overview of the research methodology.
% Suggested structure:
% - Start with requirements analysis and conceptual design.
% - Develop independent metric collectors and a timing engine.
% - Use iterative implementation with continuous integration on a real Kubernetes cluster.
% - Validate correctness through controlled experiments (planned for VT3).
%
% Keep this section short and focus on the workflow, not the technical details.

%--------------------------------------------------------------------
\section{Ethical and Sustainability Considerations}
\label{sec:intro_ethics}
% Keep this short.
% Mention:
% - Energy transparency contributes to more efficient resource use in data centers.
% - Research tools like Tycho enable more rigorous energy evaluations and can support sustainability goals.
% - No personal or sensitive data is collected; measurements are technical and device-oriented.
% - Ethical considerations primarily relate to promoting efficient cloud operation and reproducible research.

%--------------------------------------------------------------------
\section{Thesis Structure}
\label{sec:intro_structure}
% Provide a brief outline of the upcoming chapters:
% - Chapter 2: Background and related work.
% - Chapter 3: Requirements and design goals of Tycho.
% - Chapter 4: Tycho architecture and system design.
% - Later chapters: Implementation details, evaluation plan, and conclusions.
% Keep this concise.

