\chapter{Conceptual Foundations of Container-Level Power Attribution}
\label{ch:concepts}
Some of these concepts have already been discussed in  \appchapterref{A}{vt2_Chapter3}.

% This chapter introduces the conceptual principles required to understand
% workload-level energy attribution in multi-tenant Kubernetes environments.
% It provides the theoretical background for interpreting measurement behaviour,
% resource tracking, and attribution logic. It MUST NOT contain:
%   - empirical measurement characteristics (Chapter 2)
%   - Tycho design decisions (Chapter 4)
%   - Tycho implementation detail (Chapter 5)
%   - detailed review of telemetry interfaces (Chapter 2)
% Instead, it establishes the conceptual language used throughout the architecture.

%====================================================================
\section{Fundamentals of Power Attribution}
\label{sec:fundamentals_attribution}

%--------------------------------------------------------------------
\subsection{Multi-Tenant Compute Environments}
\label{subsec:multi_tenant}

% Placeholder (0.3–0.5 page):
% - Conceptually describe multi-tenant servers as environments hosting many 
%   overlapping workloads sharing physical resources.
% - Introduce the notion that hardware activity reflects aggregated behaviour.
% - DO NOT reference specific telemetry sources (RAPL, NVML, etc.).
% - DO NOT cite empirical update intervals or measurement behaviour.

%--------------------------------------------------------------------
\subsection{Meaning and Purpose of Attribution}
\label{subsec:meaning_attribution}

% Placeholder (0.3–0.5 page):
% - Define attribution as the conceptual assignment of parts of total system
%   power/energy to individual workloads.
% - Clarify that attribution is a *logical model*, not a direct observation.
% - Explain what attribution enables: accountability, optimisation, research.
% - DO NOT discuss any specific attribution technique or implementation.

%--------------------------------------------------------------------
\subsection{Resource Use and Power Consumption}
\label{subsec:resource_power_relation}

% Placeholder (0.5 page):
% - Conceptually describe how computation, memory access, and I/O 
%   induce dynamic power consumption.
% - Emphasise that mappings between resource use and power are:
%     * non-linear
%     * workload-dependent
%     * hardware-dependent
% - DO NOT present measurement technologies (these belong to Chapter 2).
% - DO NOT discuss model formulas or Tycho mechanics.

%--------------------------------------------------------------------
\subsection{Assumptions and Limits of Attribution Models}
\label{subsec:attribution_assumptions}

% Placeholder (0.4–0.6 page):
% - Explain foundational assumptions:
%     * node power represents aggregated subsystem activity
%     * workloads cannot be physically isolated perfectly
%     * attribution depends on correlations rather than causality
% - Discuss conceptual limits like non-uniqueness and unavoidable uncertainty.
% - DO NOT mention known empirical issues (idle noise, timestamps, etc.—Chapter 2).

%--------------------------------------------------------------------
\subsection{Power and Energy}
\label{subsec:power_vs_energy}

% Placeholder (0.3–0.4 page):
% - Define power (rate) and energy (integral).
% - Explain concept of attributing energy over windows rather than instantaneous power.
% - Keep fully conceptual — no references to powercap, RAPL counters, etc.
% - This prepares understanding for integration windows in Chapter 4.

%====================================================================
\section{Execution and Resource Tracking in Linux and Kubernetes}
\label{sec:execution_tracking}

% This section establishes conceptual building blocks for the metadata layer.
% It MUST NOT include empirical measurement details from Chapter 2.

%--------------------------------------------------------------------
\subsection{Processes, Threads, and Scheduling Units}
\label{subsec:processes_threads}

% Placeholder (0.4–0.5 page):
% - Describe processes and threads as basic execution units.
% - Explain conceptual CPU time accumulation and context switching.
% - Clarify behaviour of short-lived tasks at a conceptual level.
% - DO NOT describe /proc, jiffies, perf, or any specific Linux API.

%--------------------------------------------------------------------
\subsection{Control Groups as Resource Boundaries}
\label{subsec:cgroups}

% Placeholder (0.4–0.5 page):
% - Introduce cgroups conceptually as hierarchical boundaries for grouping processes.
% - Emphasise conceptual purpose: accounting, aggregation, isolation.
% - DO NOT describe cgroup file systems, versions, or metrics (kept abstract).

%--------------------------------------------------------------------
\subsection{Resource Usage Counters}
\label{subsec:linux_counters}

% Placeholder (0.3–0.5 page):
% - Introduce conceptual categories of resource counters:
%     * CPU time
%     * memory usage
%     * block I/O
%     * network use
% - Emphasise that counters update asynchronously and accumulate state.
% - DO NOT mention specific Linux files, kernel API, or numeric behaviour.
% - DO NOT reference any telemetry discussed in Chapter 2.

%--------------------------------------------------------------------
\subsection{Kubernetes Abstractions}
\label{subsec:kubernetes_abstractions}

% Placeholder (0.4–0.5 page):
% - Conceptually describe pods, containers, init containers, pause containers.
% - Explain the mapping: pod → container → processes → cgroups.
% - Keep fully conceptual; no details of kubelet API, cadvisor, or metadata sources.

%--------------------------------------------------------------------
\subsection{Lifecycle Semantics Relevant for Attribution}
\label{subsec:lifecycle_semantics}

% Placeholder (0.5–0.7 page):
% - Conceptually describe workload events:
%     * container start/stop
%     * restart
%     * crash
%     * ephemeral execution
% - Emphasise that lifetimes may overlap and be very short.
% - DO NOT mention empirical behaviour of kubelet, cAdvisor, or runtimes (Chapter 2).

%====================================================================
\section{Temporal and Measurement Concepts}
\label{sec:temporal_fundamentals}

% This section merges conceptual timing theory and measurement logic.
% It MUST NOT state empirical update intervals or sensor behaviours (Chapter 2).

%--------------------------------------------------------------------
\subsection{Sampling versus Event-Time}
\label{subsec:sampling_event_time}

% Placeholder (0.4–0.6 page):
% - Conceptually contrast periodic sampling with event-driven changes.
% - Describe mismatch: system changes between samples.
% - DO NOT describe specific telemetry source behaviour (Chapter 2).

%--------------------------------------------------------------------
\subsection{Clock Models}
\label{subsec:clock_models}

% Placeholder (0.4–0.6 page):
% - Explain realtime vs monotonic clocks conceptually.
% - Describe why monotonic clocks are preferred for temporal ordering.
% - DO NOT mention Tycho's clock.Mono implementation.

%--------------------------------------------------------------------
\subsection{Heterogeneous Metric Sources}
\label{subsec:heterogeneous_sources}

% Placeholder (0.4 page):
% - Describe categories conceptually:
%     * hardware counters
%     * OS counters
%     * device-level telemetry
%     * external interfaces
% - DO NOT repeat characteristics from Chapter 2; only conceptual classification.

%--------------------------------------------------------------------
\subsection{Delay, Jitter, and Asynchrony}
\label{subsec:delay_jitter}

% Placeholder (0.4–0.5 page):
% - Define conceptual timing phenomena:
%     * sampling delay
%     * update delay
%     * jitter
% - Explain high-level implications for reconstructing behaviour.
% - DO NOT give real-world values, update intervals, or empirical examples.

%--------------------------------------------------------------------
\subsection{Observation Windows and Temporal Alignment}
\label{subsec:observation_windows}

% Placeholder (0.5–0.7 page):
% - Introduce conceptual observation windows for:
%     * integrating power over time
%     * aligning heterogeneous data
% - Explain alignment challenges conceptually.
% - NO formulas, NO implementation-level detail.

%====================================================================
\section{Power and Utilisation Metrics (Conceptual View)}
\label{sec:metrics_types}

% This section MUST remain conceptual and independent of technologies.
% The detailed behaviour of RAPL, NVML, etc. is Chapter 2.

%--------------------------------------------------------------------
\subsection{CPU Activity and Utilisation}
\label{subsec:cpu_utilisation}

% Placeholder (0.3–0.4 page):
% - Conceptually describe CPU utilisation as fraction of time runnable/active.
% - Explain its use as a coarse proxy for work.
% - DO NOT discuss scheduler-specific metrics or kernel counters.

%--------------------------------------------------------------------
\subsection{Power Domains and Subsystem Behaviour}
\label{subsec:power_domains}

% Placeholder (0.6–0.8 page):
% - Introduce the concept of hardware power domains (CPU, memory, GPU, SoC).
% - Explain that each domain behaves differently and responds to different workloads.
% - DO NOT reference RAPL domain names or MIG partitioning (Chapter 2).

%--------------------------------------------------------------------
\subsection{External Measurements}
\label{subsec:external_measurements}

% Placeholder (0.3–0.4 page):
% - Conceptually describe the idea of external power interfaces.
% - Emphasise that they provide whole-system measurements.
% - DO NOT mention Redfish or its behaviour (Chapter 2).

%====================================================================
\section{Attribution Models and Philosophies}
\label{sec:attribution_models}

% This section introduces conceptual frameworks used in the literature.
% It MUST NOT describe Tycho's design or implementation.

%--------------------------------------------------------------------
\subsection{Container-Centric Attribution}
\label{subsec:container_centric}

% Placeholder (0.4 page):
% - Describe proportional assignment of dynamic power to container activity.
% - Conceptual advantages: interpretability, simplicity.
% - Conceptual limitations: ignores static power and shared costs.

%--------------------------------------------------------------------
\subsection{Shared-Cost Attribution}
\label{subsec:shared_cost}

% Placeholder (0.4–0.5 page):
% - Describe sharing approaches where system-level power is divided among active workloads.
% - Discuss conceptual trade-offs between fairness and accuracy.

%--------------------------------------------------------------------
\subsection{Residual Modelling}
\label{subsec:residual_modelling}

% Placeholder (0.4 page):
% - Introduce idea of modelling background or unassignable power as a residual term.
% - Explain conceptual necessity when workloads do not explain all power.

%--------------------------------------------------------------------
\subsection{Process Idle versus CPU Idle}
\label{subsec:idle_difference}

% Placeholder (0.4 page):
% - Conceptually distinguish process idle (app not running) from hardware idle (C-state).
% - Explain why this complicates attribution models.
% - DO NOT cite empirical idle-state behaviour (Chapter 2).

%--------------------------------------------------------------------
\subsection{Unaccounted Energy and Model Limitations}
\label{subsec:unaccounted_energy}

% Placeholder (0.4 page):
% - Describe conceptual sources of unaccounted energy:
%     * background daemons
%     * subsystem activity outside model
%     * asynchronous events
% - Emphasise that attribution is inherently approximate.

%====================================================================
\section{Domain Challenges}
\label{sec:domain_challenges}

% This section summarises general conceptual difficulties faced by any
% attribution system. Empirical expressions of these difficulties belong to Chapter 2.

%--------------------------------------------------------------------
\subsection{Asynchronous Metric Sources}
\label{subsec:async_sources}

% Placeholder (0.3–0.4 page):
% - Conceptually describe the challenge of combining asynchronous streams.

%--------------------------------------------------------------------
\subsection{Uneven Temporal Granularity}
\label{subsec:uneven_granularity}

% Placeholder (0.3–0.4 page):
% - Describe varying update/refresh characteristics conceptually.

%--------------------------------------------------------------------
\subsection{Lifecycle Volatility}
\label{subsec:lifecycle_challenges}

% Placeholder (0.3–0.4 page):
% - Emphasise that processes and containers may appear/disappear rapidly.

%--------------------------------------------------------------------
\subsection{Short-Lived Execution Units}
\label{subsec:short_lived_units}

% Placeholder (0.3 page):
% - Describe conceptual difficulty of capturing units that exist for less than a window.

%--------------------------------------------------------------------
\subsection{Multi-Domain Power Paths}
\label{subsec:multi_domain_powerpaths}

% Placeholder (0.3–0.4 page):
% - Conceptually explain that power flows span multiple subsystems.

%--------------------------------------------------------------------
\subsection{Measurement Uncertainty and Noise}
\label{subsec:noise_uncertainty}

% Placeholder (0.4 page):
% - Describe conceptual uncertainty from noise, variability, and filtering.

%====================================================================
\section{Summary}
\label{subsec:summary_ch3}

% Placeholder (0.4–0.5 page):
% - Summarise the conceptual challenges and abstractions.
% - State explicitly that Chapter 4 builds on these foundations to introduce
%   an accuracy-first architecture without relying on any of the empirical
%   details from Chapter 2.


\chapter{THIS IS JUST A PLACEHOLDER}