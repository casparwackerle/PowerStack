\chapter{Background and Related Research}
\label{ch:background}

% This chapter surveys existing research on power and energy measurement
% in modern server systems, summarizes key telemetry sources, reviews
% container-level attribution tools (with emphasis on Kepler and Kubewatt),
% and synthesizes research gaps that motivate the development of a new system.
% It deliberately excludes all Tycho-specific design decisions.

\section{Introduction and Scope}
\label{sec:background_scope}

% Placeholder (0.5 page):
% Introduce the purpose of Chapter 2.
% Explain that this chapter covers measurement technologies, related tools,
% and research findings relevant for fine-grained energy attribution.
% Clarify that the chapter does not describe Tycho or any architectural decisions.
% Position the chapter as the bridge between general research and the conceptual
% foundations that will follow in Chapter 3.

\section{Energy Measurement in Modern Server Systems}
\label{sec:background_measurement_overview}

\subsection{Energy Attribution in Multi-Tenant Environments}
\label{subsec:attribution_context}

% Placeholder (0.5 page):
% Summarize the general idea of energy attribution as discussed in prior work.
% Focus on research literature that explains why shared servers complicate attribution.
% No explanation of attribution models or Tycho-specific content.
% Mention the prevalence of multi-tenant cloud systems and container orchestration
% as motivating factors in the literature.

\subsection{Telemetry Layers in Contemporary Architectures}
\label{subsec:telemetry_layers}

% Placeholder (0.5 page):
% Introduce the distinction found in literature between in-band and out-of-band
% measurement mechanisms.
% Explain that research relies on a heterogeneous set of metric sources with
% different sampling rates, delays, and accuracy characteristics.
% Tie this to the complexity of correlating metrics, as mentioned in several studies.

\subsection{Challenges for Container-Level Measurement}
\label{subsec:container_challenges}

% Placeholder (0.5–1.0 page):
% Summarize research statements on why container-level attribution is difficult:
% - ephemeral workloads and short-lived processes
% - cgroup changes, metadata delays, lifecycle churn
% - Kubelet–runtime–cgroup inconsistencies noted in prior studies
% Avoid conceptual explanations of attribution models (reserved for Chapter 3).

\section{Power and Utilization Measurement Technologies}
\label{sec:measurement_techniques}

% This is one of the longest sections (4–6 pages total).
% Provide detailed summaries of the main telemetry sources and their research-backed
% strengths and limitations.

\subsection{Direct Hardware Measurement Approaches}
\label{subsec:direct_measurement}

% Placeholder (0.25–0.5 page):
% Explain, based on literature, that direct electrical measurement exists but is
% rarely practical.
% Emphasize commissioning cost, hardware requirements, and lack of deployability.
% Cite research that uses external meters but stresses limitations for real-world use.

\subsection{ACPI and Legacy Mechanisms}
\label{subsec:acpi_legacy}

% Placeholder (0.2 page):
% Briefly note that ACPI provides only power states, not measurable values.
% Mention IMPI as historically relevant but replaced by Redfish.

\subsection{Redfish Power Telemetry}
\label{subsec:redfish}

% Placeholder (1 page):
% - Describe the Redfish measurement model from literature.
% - Summarize update intervals (multi-second) and typical staleness.
% - Emphasize accuracy for total node power but insufficient temporal granularity.
% - Discuss vendor variability and the impact on short-duration workload studies.
% - Mention research noting complementarity with faster CPU-domain telemetry.

\subsection{RAPL: CPU and Memory Domains}
\label{subsec:rapl}

% Placeholder (1.5–2.0 pages):
% - Describe RAPL domains (package, core, uncore, dram) as per literature.
% - Explain known uncertainty about domain boundaries and hardware differences.
% - Summarize validation studies establishing modern RAPL accuracy.
% - Mention AMD’s reduced domain support and weaker research coverage.
% - Discuss lack of timestamps and implications for temporal alignment.
% - Present observed issues in idle measurement and high-frequency sampling.
% - Introduce multiple access mechanisms (MSR, powercap, perf, eBPF),
%   and note that research treats powercap as most practical and accurate.

\subsection{GPU Power and Utilization Metrics (NVML)}
\label{subsec:nvml}

% Placeholder (1–1.5 pages):
% - Describe NVML telemetry: power draw, utilization, clocks, etc.
% - Summarize research on sampling delays, internal averaging, and error margins.
% - Mention virtualization challenges (MIG) and associated domain ambiguity.
% - Emphasize consensus in literature that NVML is the best available despite limitations.

\subsection{Linux Kernel Utilization Metrics}
\label{subsec:linux_util}

% Placeholder (0.5 page):
% - Briefly describe CPU time accounting (jiffies), I/O statistics, and network counters.
% - Summarize their role as complementary metrics used in earlier research.
% - Emphasize limitations for energy estimation as described in prior studies.

\subsection{Subsystems Without Usable Telemetry: Storage, NICs, PCIe}
\label{subsec:lack_of_subsystem_telemetry}

% Placeholder (0.5–1.0 page):
% - Summarize the state of the research on missing or incomplete telemetry for NICs,
%   storage, and PCIe domains.
% - Highlight known NIC metrics (link speed, power states) and the limited dynamic range.
% - Mention NVMe CLI and its limited applicability to runtime energy estimation.
% - Clarify that literature identifies these domains as blind spots.

\subsection{Model-Based Estimation Approaches}
\label{subsec:model_based_estimation}

% Placeholder (0.5 page):
% - Summarize research on regression-based and model-based estimation.
% - Discuss limitations noted in literature, especially system-specific tuning.
% - Mention AI-based approaches only briefly, highlighting the training-data dependency.
% Keep this section neutral and short.

\section{Cross-Source Measurement Characteristics}
\label{sec:measurement_characteristics}

% This section synthesizes research on multi-source interactions:
% expected length: 1–2 pages.

\subsection{Temporal Characteristics and Sampling Behavior}
\label{subsec:temporal_behaviour}

% Placeholder (0.5–1.0 page):
% - Present research describing asynchronous update cycles of RAPL, NVML, Redfish.
% - Summarize studies reporting internal averaging and nondeterministic refresh timing.
% - Explain how lack of timestamps complicates alignment in prior tools.

\subsection{Domain Boundary Ambiguity}
\label{subsec:domain_boundary_ambiguity}

% Placeholder (0.5 page):
% - Summarize research identifying poorly documented domain boundaries in RAPL.
% - Discuss GPU-domain ambiguity under MIG.
% - Mention differences between CPU-package power and full-system power (Redfish).

\subsection{Validation Methodologies in Prior Research}
\label{subsec:validation_methods}

% Placeholder (0.5 page):
% - Outline how studies validated energy measurements (external meters, cross-domain comparison).
% - Mention limitations in these methodologies and potential error sources.

\section{Existing Tools and Related Work}
\label{sec:related_tools}

% Expected length: 3–4 pages (Kepler and Kubewatt form the bulk).

\subsection{Brief Overview of General Tools}
\label{subsec:general_tools}

% Placeholder (0.3 page):
% Scaphandre:
% - RAPL-only, no idle power modeling, useful exclusion of CPU inactive states.
% Smartwatts:
% - RAPL-driven server models; limited relevance for container-level analysis.
% Add 1–2 further tools if necessary, but keep extremely brief.

\subsection{Kepler}
\label{subsec:kepler_review}

% Placeholder (1.5–2.0 pages):
% - Outline the conceptual architecture of Kepler as described in literature.
% - Summarize its metric sources (RAPL, utilization, optional GPU).
% - Discuss its sampling strategy (multi-second windows) and implications noted in studies.
% - Outline its ratio-based attribution approach.
% - Summarize research observations of misattribution:
%   * inconsistencies in idle power distribution
%   * latency mismatch across metrics
%   * unstable behavior under short-lived workloads
%   * issues with system processes
% - Mention the emerging shift in Kepler’s upstream direction (toward ease of deployment).
% - Conclude with limitations identified in the literature and VT2.

\subsection{Kubewatt}
\label{subsec:kubewatt_review}

% Placeholder (1.0–1.5 pages):
% - Summarize Kubewatt’s validation of Kepler as documented in the thesis and literature.
% - List key findings:
%   * incorrect idle power attribution
%   * latency-induced attribution artefacts
%   * inconsistent handling of system processes
%   * observability issues for completed pods
% - Summarize Kubewatt’s improvements:
%   * explicit separation of static and dynamic power
%   * simplified and more predictable attribution
%   * fixes for timing-related issues
% - Describe limitations noted in the Kubewatt thesis:
%   * absence of certain metric domains
%   * limited metric scope
%   * lack of calibration mechanisms

\section{Identified Research Gaps}
\label{sec:research_gaps}

% This is one of the most important sections (2–3 pages).
% It must integrate findings from all previous sections while not revealing any Tycho design.

\subsection{Gap 1: Temporal Misalignment Across Measurement Sources}
\label{subsec:gap_temporal}

% Placeholder (0.5 page):
% - Synthesize literature showing misaligned update intervals and nondeterministic delays.
% - Highlight that existing tools cannot reliably align metrics.

\subsection{Gap 2: Lack of Timestamps and Sensor-Internal Averaging}
\label{subsec:gap_timestamps}

% Placeholder (0.3 page):
% - Emphasize the absence of timestamps in RAPL and NVML.
% - Mention research noting that internal averaging obscures instantaneous behavior.

\subsection{Gap 3: Domain Boundary Ambiguity and Incomplete Telemetry}
\label{subsec:gap_boundaries}

% Placeholder (0.5 page):
% - Summarize problems arising from undocumented boundaries (RAPL) and virtualization.
% - Discuss missing telemetry for NICs, storage, and PCIe.

\subsection{Gap 4: Metadata-Lifecycle Inconsistencies}
\label{subsec:gap_metadata}

% Placeholder (0.3–0.5 page):
% - Highlight research noting race conditions between workload events and telemetry.
% - Mention difficulties capturing terminated pods/processes reliably.

\subsection{Gap 5: Limitations in Idle Power Attribution}
\label{subsec:gap_idle}

% Placeholder (0.3–0.5 page):
% - Summarize studies showing incorrect idle attribution in existing tools.
% - Identify need for more reliable separation of static and dynamic consumption.

\subsection{Gap 6: Limited Support for Multi-Domain Integration}
\label{subsec:gap_multidomain}

% Placeholder (0.3 page):
% - Synthesize research noting lack of integrated CPU–GPU–system models.

\subsection{Gap 7: Lack of Calibration and Uncertainty Treatment}
\label{subsec:gap_calibration}

% Placeholder (0.3–0.5 page):
% - Mention that existing tools do not model measurement uncertainty or sensor delay.
% - Summarize literature calling for reproducible measurement calibration.

\section{Summary of Findings}
\label{sec:background_summary}

% Placeholder (0.5 page):
% Summarize the overall insights from prior research.
% Emphasize that the literature shows substantial progress, but also
% persistent issues in timing, attribution accuracy, telemetry coverage,
% and multi-domain integration.
% Prepare the reader for Chapter 3, which will introduce the conceptual
% foundations that guide the design of a new system.
