% !TEX root = ../main.tex

%----------------------------------------------------------------------------------------
% ABSTRACT PAGE
%----------------------------------------------------------------------------------------
\begin{abstract}
    \addchaptertocentry{\abstractname} % Add the abstract to the table of contents
    The growing energy demands of data centers have positioned energy efficiency as a critical concern in modern cloud computing. As containerization becomes the dominant approach for deploying scalable workloads, understanding the energy consumption of individual containers gains strategic relevance. However, accurately attributing energy usage to containerized workloads remains a complex and largely unsolved challenge due to hardware abstraction, shared resource utilization, and limited telemetry visibility.

    This thesis investigates the theoretical foundations, methodological challenges, and existing approaches to container-level energy consumption measurement. Emphasizing bare-metal Kubernetes environments, the study systematically explores system-level energy measurement techniques, the complexities of attributing node-level energy to individual containers, and the limitations of current measurement tools. Rather than developing a new estimation tool, this work provides a structured analysis of existing solutions, highlighting methodological gaps, validation challenges, and critical design considerations for future research and tool development.

    The findings offer a consolidated understanding of the technical factors influencing container energy attribution and outline practical recommendations for advancing energy transparency in containerized cloud infrastructures.

\end{abstract}
    
    %----------------------------------------------------------------------------------------
    % German ABSTRACT PAGE
    %----------------------------------------------------------------------------------------
    %\begin{extraAbstract}
    %\addchaptertocentry{\extraabstractname} % Add the abstract to the table of contents
    
    %Die Zusammenfassung entspricht einer Miniaturversion des gesamten Dokuments. Gliedere sie ähnlich: Beginne mit dem Kontext und der Motivation für das Projekt, einer kurzen Beschreibung der Methode und der verfügbaren Daten, Ihren Ergebnissen und den Schlussfolgerungen. Beschränke dich auf eine Seite!    
    %\end{extraAbstract}
    