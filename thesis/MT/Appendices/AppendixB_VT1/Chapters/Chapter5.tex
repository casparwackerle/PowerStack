% Indicate the main file. Must go at the beginning of the file.
% !TEX root = ../main.tex
\chapter{Test Results}
\label{vt1_Chapter5}

This chapter presents the results of the test procedures conducted to analyze Kepler-produced metrics. Each section corresponds to a specific resource type with further division into container-level and node-level metrics. The results are discussed alongside their respective figures, which illustrate Kepler-deduced energy-consumption and performance trends.

It is important to note that all Kepler metrics exhibit strong oscillations. A closer analysis shows that these oscillations follow a highly regular pattern, suggesting an issue with either Kepler’s metric publication intervals or the Prometheus scraping intervals. The data has been analyzed as-is, with moving averages added to improve readability. The implications of this irregularity will be discussed further in \hyperref[vt1_Chapter6]{Chapter~\ref*{vt1_Chapter6}}.

For clarity and to avoid confusion, three Kepler metric concepts are reiterated before discussing the results:
\begin{itemize}
\item \textbf{Package metrics:} Metrics representing the entire CPU package, including all cores and uncore components.
\item \textbf{Platform metrics:} Metrics representing the entire node.
\item \textbf{\textit{Other} metrics:} Metrics capturing platform components other than the CPU package and DRAM.
\end{itemize}

\section{CPU Stress Test Results}

\subsection{Container-Level Metrics During a CPU Stress Test}

A set of figures illustrating cache misses, CPU cycles, and CPU instructions during testing is provided in Figures~\ref{vt1_fig:cpu_Kepler_container_cache_miss_total} to~\ref{vt1_fig:cpu_Kepler_container_cpu_cycles_total}.

\begin{figure}[H]
    \centering
    \begin{subfigure}{0.49\textwidth}
        \includegraphics[width=\textwidth]{Appendices/AppendixB_VT1/Figures/diagrams/cpu/kepler_container_cache_miss_total/cpu_kepler_container_cache_miss_total_smoothed.png}
        \caption{Cache misses}
        \label{vt1_fig:cpu_Kepler_container_cache_miss_total}
    \end{subfigure}
    \begin{subfigure}{0.49\textwidth}
        \includegraphics[width=\textwidth]{Appendices/AppendixB_VT1/Figures/diagrams/cpu/kepler_container_cpu_instructions_total/cpu_kepler_container_cpu_instructions_total_smoothed.png}
        \caption{CPU instructions}
        \label{vt1_fig:cpu_Kepler_container_cpu_instructions_total}
    \end{subfigure}
    \begin{subfigure}{1\textwidth}
        \includegraphics[width=\textwidth]{Appendices/AppendixB_VT1/Figures/diagrams/cpu/kepler_container_cpu_cycles_total/cpu_kepler_container_cpu_cycles_total_smoothed.png}
        \caption{CPU cycles}
        \label{vt1_fig:cpu_Kepler_container_cpu_cycles_total}
    \end{subfigure}
    \caption[Container-Level CPU Metrics]{Kepler container-level CPU metrics during a CPU stress test}
\end{figure}

These figures illustrate cache misses, CPU cycles, and CPU instructions for the test container during execution. The diagrams show uniform trends, as the three metrics directly reflect the workload generated by \texttt{stress-ng}, which is designed to be consistent and stable. The strong correlation between the applied workload and the metrics (cache misses, CPU cycles, and CPU instructions) confirms the correct execution of the test.

Because the CPU workload running on the rest of the cluster should not affect the test container’s workload, the metric values under idle and busy cluster conditions are expected to be identical. This is indeed the case, confirming that the testing procedure was executed correctly.

A figure illustrating Kepler’s deduced Package energy consumption is provided in Figure~\ref{vt1_fig:cpu_Kepler_container_package_joules_total}.

\begin{figure}[H]
    \centering
    \includegraphics[width=1\textwidth]{Appendices/AppendixB_VT1/Figures/diagrams/cpu/kepler_container_package_joules_total/cpu_kepler_container_package_joules_total_smoothed.png}
    \caption[Container Package energy]{Kepler container-level Package energy consumption}
    \label{vt1_fig:cpu_Kepler_container_package_joules_total}
\end{figure}
    
The figure shows a clear upward trend in Package energy consumption, with distinct steps that correspond to the expected workload increases. A strong correlation is observed between Kepler’s reported Package energy consumption and the test workload. However, the relationship between energy consumption and workload is non-linear: while a 10\% workload averages around 2.5~W, a 90\% workload results in only about twice the energy consumption, despite the workload increasing by a factor of nine.

Furthermore, Kepler’s Package energy measurements remain consistent regardless of whether the node is idle or busy, showing no statistically significant difference.

A set of figures illustrating total container energy consumption, DRAM energy consumption, and \textit{Other} energy consumption components is provided in Figures~\ref{vt1_fig:cpu_Kepler_container_joules_total} to~\ref{vt1_fig:cpu_Kepler_container_dram_joules_total}.

\begin{figure}[H]
    \centering
    \begin{subfigure}{1\textwidth}
        \includegraphics[width=\textwidth]{Appendices/AppendixB_VT1/Figures/diagrams/cpu/kepler_container_joules_total/cpu_kepler_container_joules_total_smoothed.png}
        \caption{Total container-level energy consumption}
        \label{vt1_fig:cpu_Kepler_container_joules_total}
    \end{subfigure}
    \begin{subfigure}{0.49\textwidth}
        \includegraphics[width=\textwidth]{Appendices/AppendixB_VT1/Figures/diagrams/cpu/kepler_container_other_joules_total/cpu_kepler_container_other_joules_total_smoothed.png}
        \caption{\textit{Other} energy consumption}
        \label{vt1_fig:cpu_Kepler_container_other_joules_total}
    \end{subfigure}
    \begin{subfigure}{0.49\textwidth}
        \includegraphics[width=\textwidth]{Appendices/AppendixB_VT1/Figures/diagrams/cpu/kepler_container_dram_joules_total/cpu_kepler_container_dram_joules_total_smoothed.png}
        \caption{DRAM energy consumption}
        \label{vt1_fig:cpu_Kepler_container_dram_joules_total}
    \end{subfigure}
    \caption[Container-Level Energy Consumption]{Kepler container-level energy consumption during a CPU stress test}
\end{figure}

The figures for container energy consumption, DRAM energy consumption, and \textit{Other} components (representing host components excluding CPU and DRAM) show a less direct correlation with workload than the Package energy.

The main observations are:

\begin{itemize}
    \item In Figure~\ref{vt1_fig:cpu_Kepler_container_joules_total}, a slight upward trend is visible: total container energy consumption increases by roughly 5~W. This change mirrors the increase in Package energy consumption in Figure~\ref{vt1_fig:cpu_Kepler_container_package_joules_total}, where an approximate 5~W increase can also be observed.
    \item In Figure~\ref{vt1_fig:cpu_Kepler_container_other_joules_total}, which shows \textit{Other} (non-CPU/DRAM) container energy consumption, no clear trend can be identified. This is expected, since only the CPU was explicitly stressed. However, the overall magnitude of \textit{Other} energy consumption is surprisingly high, reaching roughly twice the CPU Package energy.
    \item The measured DRAM energy consumption is largely unaffected by CPU stress, as expected. With values between 0.5 and 1~W, DRAM energy remains comparatively low.
    \item During the second part of the experiment (the busy-node condition), all metrics appear slightly smoother, but they are neither significantly higher nor lower compared to the idle-node experiment.
\end{itemize}

\subsection{Node-Level Metrics During a CPU Stress Test}

Figures illustrating node-level package, DRAM, and \textit{Other} energy consumption are provided in Figures~\ref{vt1_fig:cpu_Kepler_node_package_joules_total} to~\ref{vt1_fig:cpu_Kepler_node_other_joules_total}. For node-level energy consumption, Kepler distinguishes between idle and dynamic power.

\begin{figure}[H]
    \centering
    \begin{subfigure}{1\textwidth}
        \includegraphics[width=\textwidth]{Appendices/AppendixB_VT1/Figures/diagrams/cpu/kepler_node_package_joules_total/cpu_kepler_node_package_joules_total_ho3.png}
        \caption{Node Package energy consumption}
        \label{vt1_fig:cpu_Kepler_node_package_joules_total}
    \end{subfigure}
    \begin{subfigure}{0.49\textwidth}
        \includegraphics[width=\textwidth]{Appendices/AppendixB_VT1/Figures/diagrams/cpu/kepler_node_dram_joules_total/cpu_kepler_node_dram_joules_total_ho3.png}
        \caption{Node DRAM energy consumption}
        \label{vt1_fig:cpu_Kepler_node_dram_joules_total}
    \end{subfigure}
    \begin{subfigure}{0.49\textwidth}
        \includegraphics[width=\textwidth]{Appendices/AppendixB_VT1/Figures/diagrams/cpu/kepler_node_other_joules_total/cpu_kepler_node_other_joules_total_ho3.png}
        \caption{Node \textit{Other} energy consumption}
        \label{vt1_fig:cpu_Kepler_node_other_joules_total}
    \end{subfigure}
    \caption[Node-Level Energy Consumption]{Kepler node-level energy consumption during a CPU stress test}
\end{figure}

The following observations can be made:

\begin{itemize}
\item The most striking observation is the relatively high idle energy consumption of the node, which is visible in all figures. While Figure~\ref{vt1_fig:cpu_Kepler_node_package_joules_total} shows increasing dynamic Package energy due to the CPU stress test, idle energy still far exceeds the dynamic component.
\item The dynamic \textit{Other} energy consumption shown in Figure~\ref{vt1_fig:cpu_Kepler_node_other_joules_total} appears to be largely independent of the CPU stress load. This further supports the conclusion that \textit{Other} system components contribute significantly to overall platform energy consumption but are generally unaffected by CPU workload.
\end{itemize}

\subsection{Overall Conclusions}

The CPU stress test results demonstrate that Kepler captures workload-dependent variations in energy consumption with reasonable accuracy. Key takeaways from the analysis include:

\begin{itemize}
    \item Kepler’s CPU Package energy measurements correlate with workload intensity, although the relationship is clearly non-linear.
    \item High idle energy consumption at the node level suggests that a substantial share of total energy use is independent of CPU workload.
    \item The \textit{Other} component’s energy consumption remains largely static with respect to workload, indicating that these components primarily contribute to baseline (idle) energy consumption rather than dynamic variations.
\end{itemize}
\section{Memory Stress Test Results}

\subsection{Container-Level Metrics During a Memory Stress Test}

Figures~\ref{vt1_fig:mem_Kepler_container_dram_joules_total} to~\ref{vt1_fig:mem_Kepler_container_joules_total} show the container-level energy consumption metrics published by Kepler during the memory stress test.

\begin{figure}[H]
    \centering
    \begin{subfigure}{0.49\textwidth}
        \includegraphics[width=\textwidth]{Appendices/AppendixB_VT1/Figures/diagrams/mem/kepler_container_dram_joules_total/mem_kepler_container_dram_joules_total_smoothed.png}
        \caption{DRAM energy consumption.}
        \label{vt1_fig:mem_Kepler_container_dram_joules_total}
    \end{subfigure}
    \begin{subfigure}{0.49\textwidth}
        \includegraphics[width=\textwidth]{Appendices/AppendixB_VT1/Figures/diagrams/mem/kepler_container_package_joules_total/mem_kepler_container_package_joules_total_smoothed.png}
        \caption{Package energy consumption.}
        \label{vt1_fig:mem_Kepler_container_package_joules_total}
    \end{subfigure}
    \begin{subfigure}{0.49\textwidth}
        \includegraphics[width=\textwidth]{Appendices/AppendixB_VT1/Figures/diagrams/mem/kepler_container_other_joules_total/mem_kepler_container_other_joules_total_smoothed.png}
        \caption{\textit{Other} energy consumption.}
        \label{vt1_fig:mem_Kepler_container_other_joules_total}
    \end{subfigure}
    \begin{subfigure}{0.49\textwidth}
        \includegraphics[width=\textwidth]{Appendices/AppendixB_VT1/Figures/diagrams/mem/kepler_container_joules_total/mem_kepler_container_joules_total_smoothed.png}
        \caption{Total container energy consumption.}
        \label{vt1_fig:mem_Kepler_container_joules_total}
    \end{subfigure}
    \caption{Container-level energy consumption during a memory stress test.}
\end{figure}

The following observations can be made:

\begin{itemize}
    \item None of the four published energy metrics correlate with the applied memory stress load. There is also no significant difference between executing the stress test on an idle versus a busy node. None of the energy metrics indicate that a memory stress test is being performed.
    \item Figure~\ref{vt1_fig:mem_Kepler_container_package_joules_total} shows an average container-level DRAM energy consumption of approximately 0.3~W. This is considerably lower than the 0.7~W measured during the CPU stress test (Figure~\ref{vt1_fig:cpu_Kepler_container_dram_joules_total}), which also exhibits a clear upward trend during higher CPU workloads.
\end{itemize}

\subsection{Node-Level Metrics During a Memory Stress Test}

Figures~\ref{vt1_fig:mem_Kepler_node_dram_joules_total} to~\ref{vt1_fig:mem_Kepler_node_other_joules_total} show the node-level idle and dynamic energy consumption metrics published by Kepler during the memory stress test.

\begin{figure}[H]
    \centering
    \begin{subfigure}{1\textwidth}
        \includegraphics[width=\textwidth]{Appendices/AppendixB_VT1/Figures/diagrams/mem/kepler_node_dram_joules_total/mem_kepler_node_dram_joules_total_ho3.png}
        \caption{DRAM energy consumption.}
        \label{vt1_fig:mem_Kepler_node_dram_joules_total}
    \end{subfigure}

    \begin{subfigure}{0.49\textwidth}
        \includegraphics[width=\textwidth]{Appendices/AppendixB_VT1/Figures/diagrams/mem/kepler_node_package_joules_total/mem_kepler_node_package_joules_total_ho3.png}
        \caption{Package energy consumption.}
        \label{vt1_fig:mem_Kepler_node_package_joules_total}
    \end{subfigure}
    \begin{subfigure}{0.49\textwidth}
        \includegraphics[width=\textwidth]{Appendices/AppendixB_VT1/Figures/diagrams/mem/kepler_node_other_joules_total/mem_kepler_node_other_joules_total_ho3.png}
        \caption{\textit{Other} energy consumption.}
        \label{vt1_fig:mem_Kepler_node_other_joules_total}
    \end{subfigure}

    \caption{Node-level energy consumption during a memory stress test.}
\end{figure}

The node-level metrics collected during the memory stress test present a similar picture to the container-level metrics. The following observations can be made:

\begin{itemize}
    \item No node-level energy consumption metric correlates with the memory stress applied during the test.
    \item All node-level energy consumption metrics exhibit significantly higher idle energy consumption than dynamic energy consumption.
\end{itemize}

\subsection{Overall Conclusions}

The following key takeaways can be derived from the memory stress test results:

\begin{itemize}
    \item The memory stress test does not indicate any capability of Kepler to reliably track memory energy consumption. None of the metrics respond to the various stimuli of the test scenario.
    \item However, the container-level DRAM metric appears to be affected by CPU stress, as observed in the CPU stress test.
\end{itemize}


\section{Disk I/O Stress Test Results}

\subsection{Container-Level Metrics During a Disk I/O Stress Test}

Figures~\ref{vt1_fig:cpu_Kepler_container_cache_miss_total} to~\ref{vt1_fig:diskIO_Kepler_container_cpu_instructions_total} show the CPU metrics published by Kepler during the Disk~I/O stress experiment.

\begin{figure}[H]
    \centering
    \begin{subfigure}{0.49\textwidth}
        \includegraphics[width=\textwidth]{Appendices/AppendixB_VT1/Figures/diagrams/diskIO/kepler_container_bpf_block_irq_total/diskIO_kepler_container_bpf_block_irq_total_smoothed.png}
        \caption{Block IRQ.}
        \label{vt1_fig:cpu_Kepler_container_cache_miss_total}
    \end{subfigure}
    \begin{subfigure}{0.49\textwidth}
        \includegraphics[width=\textwidth]{Appendices/AppendixB_VT1/Figures/diagrams/diskIO/kepler_container_cache_miss_total/diskIO_kepler_container_cache_miss_total_smoothed.png}
        \caption{Cache misses.}
        \label{vt1_fig:diskIO_Kepler_container_cache_miss_total}
    \end{subfigure}

    \begin{subfigure}{0.49\textwidth}
        \includegraphics[width=\textwidth]{Appendices/AppendixB_VT1/Figures/diagrams/diskIO/kepler_container_cpu_cycles_total/diskIO_kepler_container_cpu_cycles_total_smoothed.png}
        \caption{CPU instructions.}
        \label{vt1_fig:diskIO_Kepler_container_cpu_cycles_total}
    \end{subfigure}
    \begin{subfigure}{0.49\textwidth}
        \includegraphics[width=\textwidth]{Appendices/AppendixB_VT1/Figures/diagrams/diskIO/kepler_container_cpu_instructions_total/diskIO_kepler_container_cpu_instructions_total_smoothed.png}
        \caption{CPU cycles.}
        \label{vt1_fig:diskIO_Kepler_container_cpu_instructions_total}
    \end{subfigure}

    \caption{Container-level CPU metrics during a Disk~I/O stress test.}
\end{figure}

All four figures behave as expected and validate the general test procedure. Notably, the operations per second for cache misses, CPU instructions, and CPU cycles do not scale linearly with the workload. The relative difference between the low-workload and high-workload tests is approximately 330\% (cache misses), 250\% (CPU instructions), and 350\% (CPU cycles), all significantly lower than the 900\% relative difference in the applied load.

\begin{figure}[H]
    \centering
    \begin{subfigure}{0.49\textwidth}
        \includegraphics[width=\textwidth]{Appendices/AppendixB_VT1/Figures/diagrams/diskIO/kepler_container_package_joules_total/diskIO_kepler_container_package_joules_total_smoothed.png}
        \caption{Package energy consumption.}
        \label{vt1_fig:diskIO_Kepler_container_package_joules_total}
    \end{subfigure}
    \begin{subfigure}{0.49\textwidth}
        \includegraphics[width=\textwidth]{Appendices/AppendixB_VT1/Figures/diagrams/diskIO/kepler_container_dram_joules_total/diskIO_kepler_container_dram_joules_total_smoothed.png}
        \caption{DRAM energy consumption.}
        \label{vt1_fig:diskIO_Kepler_container_dram_joules_total}
    \end{subfigure}

    \begin{subfigure}{0.49\textwidth}
        \includegraphics[width=\textwidth]{Appendices/AppendixB_VT1/Figures/diagrams/diskIO/kepler_container_other_joules_total/diskIO_kepler_container_other_joules_total_smoothed.png}
        \caption{\textit{Other} energy consumption.}
        \label{vt1_fig:diskIO_Kepler_container_other_joules_total}
    \end{subfigure}
    \begin{subfigure}{0.49\textwidth}
        \includegraphics[width=\textwidth]{Appendices/AppendixB_VT1/Figures/diagrams/diskIO/kepler_container_joules_total/diskIO_kepler_container_joules_total_smoothed.png}
        \caption{Total container energy consumption.}
        \label{vt1_fig:diskIO_Kepler_container_joules_total}
    \end{subfigure}

    \caption{Container-level energy consumption during a Disk~I/O stress test.}
\end{figure}

Figures~\ref{vt1_fig:diskIO_Kepler_container_package_joules_total} to~\ref{vt1_fig:diskIO_Kepler_container_joules_total} show the Kepler-reported energy metrics for the package, DRAM, \textit{Other} components, and total container energy consumption. Several observations can be made:

\begin{itemize}
    \item All four metrics exhibit a bimodal distribution, oscillating between distinct \emph{low} and \emph{high} states. While varying HDD rotation speeds might explain the curves observed for \textit{Other} components and total container energy consumption, this does not account for the behaviour observed in package and DRAM energy consumption.
    \item The relative energy consumption across all four metrics appears nearly identical regardless of the component measured. This is particularly evident for the \textit{Other} components and total container energy consumption, which appear identical aside from a scaling factor of approximately 1.5.
    \item All figures show strong temporal correlation with the test intervals, indicating that Kepler does detect changes in disk load.
\end{itemize}

To further analyze the metrics during a Disk~I/O stress test, the experiment was repeated; the container-level Package energy consumption is shown in Figure~\ref{vt1_fig:diskIO_Kepler_container_cpu_joules_total_second_experiment}.

\begin{figure}[H]
    \centering
    \includegraphics[width=0.49\textwidth]{Appendices/AppendixB_VT1/Figures/diagrams/diskIO2/kepler_container_package_joules_total/diskIO_kepler_container_package_joules_total_smoothed.png}
    \caption[Container Package energy]{Container-level Package energy consumption during a second experiment.}
    \label{vt1_fig:diskIO_Kepler_container_cpu_joules_total_second_experiment}
\end{figure}

The following observations were consistent with the first experiment:

\begin{itemize}
    \item All joule-based metrics still displayed a bimodal distribution.
    \item All joule-based metrics appear identical aside from differences in scale.
    \item All joule-based metrics show an observable upward trend.
\end{itemize}

\subsection{Node-Level Metrics During a Disk I/O Stress Test}

Figures~\ref{vt1_fig:diskIO_Kepler_node_package_joules_total} to~\ref{vt1_fig:diskIO_Kepler_node_other_joules_total} show the node-level energy consumption for the package, DRAM, and \textit{Other} components, split into idle and dynamic energy consumption.

\begin{figure}[H]
    \centering
    \begin{subfigure}{0.7\textwidth}
        \includegraphics[width=\textwidth]{Appendices/AppendixB_VT1/Figures/diagrams/diskIO/kepler_node_package_joules_total/diskIO_kepler_node_package_joules_total_ho3.png}
        \caption{Node Package energy consumption.}
        \label{vt1_fig:diskIO_Kepler_node_package_joules_total}
    \end{subfigure}

    \begin{subfigure}{0.49\textwidth}
        \includegraphics[width=\textwidth]{Appendices/AppendixB_VT1/Figures/diagrams/diskIO/kepler_node_dram_joules_total/diskIO_kepler_node_dram_joules_total_ho3.png}
        \caption{Node DRAM energy consumption.}
        \label{vt1_fig:diskIO_Kepler_node_dram_joules_total}
    \end{subfigure}
    \begin{subfigure}{0.49\textwidth}
        \includegraphics[width=\textwidth]{Appendices/AppendixB_VT1/Figures/diagrams/diskIO/kepler_node_other_joules_total/diskIO_kepler_node_other_joules_total_ho3.png}
        \caption{Node \textit{Other} energy consumption.}
        \label{vt1_fig:diskIO_Kepler_node_other_joules_total}
    \end{subfigure}

    \caption[Node-Level Energy Consumption]{Kepler node-level energy consumption during a Disk~I/O stress test.}
\end{figure}

The following observations can be made regarding node-level energy consumption:

\begin{itemize}
    \item Across all components, the difference between idle and dynamic power is pronounced. The majority of node energy consumption occurs as idle power.
    \item The node DRAM and \textit{Other} components do not show an increasing trend in dynamic energy consumption. In contrast, Figure~\ref{vt1_fig:diskIO_Kepler_node_package_joules_total} shows an upward trend of approximately 50\% in dynamic Package energy consumption, although this remains significantly overshadowed by idle energy consumption.
\end{itemize}
\subsection{Overall Conclusions}

The following overall conclusions can be drawn regarding Kepler metrics during a Disk~I/O test:

\begin{itemize}
    \item Kepler provides plausible metrics for Block IRQ, cache misses, CPU instructions, and CPU cycles during a Disk~I/O stress test.
    \item The accuracy of Kepler’s container-level joule-based metrics is questionable. While they exhibit an intriguing bimodal distribution, they do not necessarily align with the Disk~I/O workload.
    \item Kepler’s container-level joule-based metrics reliably \emph{detect} changes in Disk~I/O workload, but their predictability remains uncertain.
    \item Node-level Kepler metrics do not suggest any significant impact of Disk~I/O workload on DRAM or \emph{Other} components.
    \item The fact that the variations in energy consumption shown in container-level metrics cannot clearly be traced in either idle or dynamic node energy consumption raises further questions.
    \item The hypothesis that Disk~I/O stress does not significantly contribute to node power cannot be rejected or proven. Further research is necessary.
\end{itemize}

\section{Network I/O Stress Test Results}

\subsection{Container-Level Metrics During a Network I/O Stress Test}

Figures~\ref{vt1_fig:netIO_Kepler_container_cache_miss_total} to~\ref{vt1_fig:netIO_Kepler_container_cpu_instructions_total} show the CPU metrics published by Kepler during the Network~I/O stress experiment.

\begin{figure}[H]
    \centering
    \begin{subfigure}{1\textwidth}
        \includegraphics[width=\textwidth]{Appendices/AppendixB_VT1/Figures/diagrams/netIO/kepler_container_cache_miss_total/netIO_kepler_container_cache_miss_total_smoothed.png}
        \caption{Cache misses.}
        \label{vt1_fig:netIO_Kepler_container_cache_miss_total}
    \end{subfigure}

    \begin{subfigure}{0.49\textwidth}
        \includegraphics[width=\textwidth]{Appendices/AppendixB_VT1/Figures/diagrams/netIO/kepler_container_cpu_cycles_total/netIO_kepler_container_cpu_cycles_total_smoothed.png}
        \caption{CPU instructions.}
        \label{vt1_fig:netIO_Kepler_container_cpu_cycles_total}
    \end{subfigure}
    \begin{subfigure}{0.49\textwidth}
        \includegraphics[width=\textwidth]{Appendices/AppendixB_VT1/Figures/diagrams/netIO/kepler_container_cpu_instructions_total/netIO_kepler_container_cpu_instructions_total_smoothed.png}
        \caption{CPU cycles.}
        \label{vt1_fig:netIO_Kepler_container_cpu_instructions_total}
    \end{subfigure}

    \caption{Container-level CPU metrics during a Network~I/O stress test.}
\end{figure}

The results do not provide a clear explanation for the applied test stress. While the metrics correlate in time with the applied Network~I/O load, their values do not clearly correspond to the applied stress. The Kepler metrics for RX IRQ and TX IRQ in Figures~\ref{vt1_fig:netIO_Kepler_container_bpf_net_rx_irq_total} and~\ref{vt1_fig:netIO_Kepler_container_bpf_net_tx_irq_total} present a similar trend, though with significantly higher variance.

\begin{figure}[H]
    \centering
    \begin{subfigure}{0.49\textwidth}
        \includegraphics[width=\textwidth]{Appendices/AppendixB_VT1/Figures/diagrams/netIO/kepler_container_bpf_net_rx_irq_total/netIO_kepler_container_bpf_net_rx_irq_total_smoothed.png}
        \caption{RX IRQ.}
        \label{vt1_fig:netIO_Kepler_container_bpf_net_rx_irq_total}
    \end{subfigure}
    \begin{subfigure}{0.49\textwidth}
        \includegraphics[width=\textwidth]{Appendices/AppendixB_VT1/Figures/diagrams/netIO/kepler_container_bpf_net_tx_irq_total/netIO_kepler_container_bpf_net_tx_irq_total_smoothed.png}
        \caption{TX IRQ.}
        \label{vt1_fig:netIO_Kepler_container_bpf_net_tx_irq_total}
    \end{subfigure}
    \caption{Container-level IRQ metrics during a Network~I/O stress test.}
\end{figure}

\begin{figure}[H]
    \centering
    \begin{subfigure}{0.49\textwidth}
        \includegraphics[width=\textwidth]{Appendices/AppendixB_VT1/Figures/diagrams/netIO/kepler_container_package_joules_total/netIO_kepler_container_package_joules_total_smoothed.png}
        \caption{Package energy consumption.}
        \label{vt1_fig:netIO_Kepler_container_package_joules_total}
    \end{subfigure}
    \begin{subfigure}{0.49\textwidth}
        \includegraphics[width=\textwidth]{Appendices/AppendixB_VT1/Figures/diagrams/netIO/kepler_container_dram_joules_total/netIO_kepler_container_dram_joules_total_smoothed.png}
        \caption{DRAM energy consumption.}
        \label{vt1_fig:netIO_Kepler_container_dram_joules_total}
    \end{subfigure}

    \begin{subfigure}{0.49\textwidth}
        \includegraphics[width=\textwidth]{Appendices/AppendixB_VT1/Figures/diagrams/netIO/kepler_container_other_joules_total/netIO_kepler_container_other_joules_total_smoothed.png}
        \caption{\emph{Other} energy consumption.}
        \label{vt1_fig:netIO_Kepler_container_other_joules_total}
    \end{subfigure}
    \begin{subfigure}{0.49\textwidth}
        \includegraphics[width=\textwidth]{Appendices/AppendixB_VT1/Figures/diagrams/netIO/kepler_container_joules_total/netIO_kepler_container_joules_total_smoothed.png}
        \caption{Total container energy consumption.}
        \label{vt1_fig:netIO_Kepler_container_joules_total}
    \end{subfigure}

    \caption{Container-level energy consumption during a Network~I/O stress test.}
\end{figure}

Figures~\ref{vt1_fig:netIO_Kepler_container_package_joules_total} to~\ref{vt1_fig:netIO_Kepler_container_joules_total} display Kepler metrics for package, DRAM, \emph{Other}, and overall container-level energy consumption. The patterns remain consistent: the metrics indicate transitions between different workloads, but the values (while nearly constant for each experiment) appear unrelated to the Network~I/O activity.

\subsection{Node-Level Metrics During a Network I/O Stress Test}

Figures~\ref{vt1_fig:netIO_Kepler_node_dram_joules_total} to~\ref{vt1_fig:netIO_Kepler_node_other_joules_total} show the node-level idle and dynamic energy consumption metrics published by Kepler during the Network~I/O stress test.

\begin{figure}[H]
    \centering
    \begin{subfigure}{0.49\textwidth}
        \includegraphics[width=\textwidth]{Appendices/AppendixB_VT1/Figures/diagrams/netIO/kepler_node_dram_joules_total/netIO_kepler_node_dram_joules_total_ho3.png}
        \caption{DRAM energy consumption.}
        \label{vt1_fig:netIO_Kepler_node_dram_joules_total}
    \end{subfigure}
    \begin{subfigure}{0.49\textwidth}
        \includegraphics[width=\textwidth]{Appendices/AppendixB_VT1/Figures/diagrams/netIO/kepler_node_package_joules_total/netIO_kepler_node_package_joules_total_ho3.png}
        \caption{Package energy consumption.}
        \label{vt1_fig:netIO_Kepler_node_package_joules_total}
    \end{subfigure}
    \begin{subfigure}{0.49\textwidth}
        \includegraphics[width=\textwidth]{Appendices/AppendixB_VT1/Figures/diagrams/netIO/kepler_node_other_joules_total/netIO_kepler_node_other_joules_total_ho3.png}
        \caption{\emph{Other} energy consumption.}
        \label{vt1_fig:netIO_Kepler_node_other_joules_total}
    \end{subfigure}
    \caption{Node-level energy consumption during a Network~I/O stress test.}
\end{figure}

Much like the previous Disk~I/O test, the node-level energy consumption metrics do not show any clear indication of the applied network load. The only notable observation is the stepwise increase in dynamic Package energy consumption, which is likely related to the actual generation of network traffic.

\subsection{Overall Conclusions}

Unfortunately, the test does not indicate a clear relationship between the applied Network~I/O load and the Kepler metrics. Similar to the Disk~I/O test, the following conclusions can be drawn:

\begin{itemize}
    \item The Kepler metrics are affected by the Network~I/O stress test, as evidenced by their temporal correlation with the applied workload.
    \item However, the metric values do not logically correlate with the actual network traffic being generated.
\end{itemize}

This test was repeated multiple times with consistent results: the workload continued to correlate in time with the test steps, but the metric values plateaued at seemingly arbitrary levels for the duration of each load phase.
