% !TEX root = ../main.tex

%----------------------------------------------------------------------------------------
% ABSTRACT PAGE
%----------------------------------------------------------------------------------------
\begin{abstract}
\addchaptertocentry{\abstractname} % Add the abstract to the table of contents
Accurately attributing energy consumption to individual workloads in containerized environments is challenging due to shared hardware resources, limited observability, and asynchronous, heterogeneous telemetry. This thesis presents \emph{Tycho}, a novel, accuracy-first system for container-level power attribution that departs from window-delta-based approaches by retaining high-frequency measurement data in bounded historical buffers and deferring temporal reconciliation to analysis time. Independent metric collectors operate at source-appropriate frequencies, preserving native temporal structure and enabling post hoc alignment and correlation across diverse energy and utilization signals.

Building on an extensive review of existing power measurement and attribution approaches, the thesis develops a principled framework that explicitly models observation delay, enforces energy conservation by construction, and treats idle and residual energy as first-class outcomes. Tycho integrates CPU, GPU, and system-level energy signals, including composite GPU energy modelling from heterogeneous telemetry and delay-aware refinement of coarse system energy measurements using fine-grained subsystem proxies. The system is evaluated qualitatively and quantitatively on representative and concurrent workloads, demonstrating accurate and temporally coherent attribution behaviour across diverse execution scenarios without implying a unique ground truth.

Tycho is released as an open-source contribution to support reproducibility and further research in energy-aware systems \parencite{TychoRepo}. Experimental deployment and evaluation are supported by PowerStack, an auxiliary framework for fully automated installation and test environment setup \parencite{PowerStack}.

\end{abstract}
    
    %----------------------------------------------------------------------------------------
    % German ABSTRACT PAGE
    %----------------------------------------------------------------------------------------
    %\begin{extraAbstract}
    %\addchaptertocentry{\extraabstractname} % Add the abstract to the table of contents
    
    %Die Zusammenfassung entspricht einer Miniaturversion des gesamten Dokuments. Gliedere sie ähnlich: Beginne mit dem Kontext und der Motivation für das Projekt, einer kurzen Beschreibung der Methode und der verfügbaren Daten, Ihren Ergebnissen und den Schlussfolgerungen. Beschränke dich auf eine Seite!    
    %\end{extraAbstract}
    