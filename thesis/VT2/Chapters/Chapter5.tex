\chapter{Designing a Container Power Attribution Architecture} % Main chapter title
\label{Chapter5}


    5.1 Design Goals
        Generalizability, minimal overhead, real-time capability, etc.
    5.2 Architecture Components
        Metric sources, data aggregation, correlation layer, exporter
    5.3 Attribution Logic
        CPU (RAPL + cgroups), RAM, NET, DISK, idle, system
    5.4 Proposed Correlation Model
        Hybrid models (e.g. direct for CPU, proportional for network)
    5.5 Open Questions and Future Improvements


Notes
    VM nesting (like scaphandre) in Kepler -> passthrough
    explicitely handle PID0, idle task
    let the user define interal measurement intervals
        kepler exporter: multiple of measurement intervals.
    fundamentally: SHOULD we cross-reference node and process consumption, with drastically different accuracies?
    did kepler just stop midway? no activity in the github, but many TODO, currently 1053.


    just write a word on the interchangeable words power and energyy

    kubewatt base initialization is good, but could this be better combined in a continuous measurement like kepler does? overhead issue certainly not?
    also Kubewatts expects users to specify which pod names are part of the control plane -> not really fire-and-forget