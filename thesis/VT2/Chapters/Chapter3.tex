\chapter{Analysis of Energy Consumption Tools} % Main chapter title
\label{Chapter3}


\section{Introduction}
    Explain categories:
        General Server Monitoring
        Container-Level Monitoring

\section{General Server Monitoring Tools}
    PowerSensor3, Powertop, Green Metrics Tool, Kavanagh 2019.
    Focus on system-wide measurement.
    Capabilities and limitations.

\section{Container-Level Monitoring Tools}
    KEPLER, Scaphandre, Smartwatts, JoularJX, AI Power Meter, CodeCarbon.
    Granularity down to the container level.
    internal mechanisms (e.g., eBPF, RAPL, NVML).
    Advantages and drawbacks.

\section{Comparison of Tools}
    Detailed matrix comparing:
        Measurement methodology.
        Component focus (CPU, RAM, GPU, Disk, Network).
        Real-time capabilities.
        Kubernetes compatibility.

\section{Energy Attribution Techniques in containerized environments}
    introduction. the problem is non-trivial
    usage-based, event-based, statistical modelling, ...
    where to put the system usage

\subsection{Data fusion techniques}
    resource usage correlation (CPU time, Mem, I/O)
    time based attribution
    event-driven attribution

\subsection{multi-tenant complexity}
    challenges of multi-tenant workloads
    isolation issues, nosy neighbors, contention

\subsection{Approaches to solve attribution}

\subsection{Evaluation of current techniques}





-----------------
\section{Tools}
\subsection{RAPL-based tools}
\label{sec:rapltools}
\begin{itemize}
    \item \parencite{jay2023experimental} An experimental comparison of software-based power meters (focus on CPU / GPU)
    \item \parencite{van2025powersensor3} fast accurate opensource: PowerSensor3 enables real-time power measurements of SoC boards and PCIe cards, including GPUs, FPGAs, NICs, SSDs, and domain-specific AI and ML accelerators
    \item \parencite{kavanagh2019rapid} Rapid and accurate energy models through calibration with IPMI and RAPL
    \item \parencite{scaphandre_documentation} Scaphandre. Does not handle overflows correctly (https://github.com/hubblo-org/scaphandre/issues/280)
    \item \parencite{fieni2020smartwatts} Smartwatts: Self-Calibrating Software-Defined Power Meter for containers
    \item \parencite{joularjx} JoularJX: jaba-based agent for power monitoring at the code level
    \item \parencite{kepler_energy}: KEPLER
    \item \parencite{aipowermeter}: "AI power meter": Library to measure energy usage of machine learning programs, uses RAPL for CPU and nvidia-smi for GPU
    \item \parencite{codecarbon} CodeCarbon: Python package, estimates GPU + CPU + RAM: uses pynvml, ram RATIO (3W for 8G) and RAPL. According to Raffin2024, this tool does not account for the MSR overflow: https://github.com/mlco2/codecarbon/issues/322 -> apparently fixed now
    \item \parencite{powertop}: powertop
    \item \parencite{greencodingdocs}: Green metrics tool: measuring energy and CO2 consumption of software through a software life cycle anslysis (SLCA): Metric providers: RAPL, IPMI, PSU, Docker, Temperature, CPU, ... (sone external devices)
    
    according to raffin2024: simplified versions of scaphandre and codecarbon hhve 3\%, 0.5\% overhead at 10Hz
    according to \parencite{jay2023experimental}, the full versions have between 2 and 7\% at 1Hz.

\parencite{fieni2024powerapi}: PowerAPI: Python framework for building software-defined power
\end{itemize}
\begin{comment}
- multiple papers have tried to attribute component-level 
\end{comment}




\section{"data fusion of power data and cpu metrics}
- Estimating the consumption of a single function has been proven to be possible in 2012: M. Hähnel, B. Döbel, M. Völp, and H. Härtig, “Measuring energy consumption for short code paths using RAPL,”, \parencite{hahnel2012measuring}











Solutions that would improve the general situation
- homogeneous servers, making more specific analysis financially viable
- more research of course
- 