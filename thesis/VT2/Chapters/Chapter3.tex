\chapter{Correlating Power Consumption to Containers} % Main chapter title
\label{Chapter3}


Chapter 3: Correlating Power Consumption to Containers
    3.1 The Challenge of Power Attribution in Shared Systems
    3.2 What Makes a Good Correlation?
        Accuracy
        Temporal granularity
        Causality vs. correlation
        Real-time vs. post-processing tradeoffs
    3.3 Power Attribution Philosophies
        Usage-based, proportional, fixed-share, hybrid
        Dealing with idle power
        Dealing with system containers / background processes
    3.4 Contextual Use Cases
        Cost/billing
        Sustainability / carbon reporting
        Resource optimization
        SLA verification / fairness in multi-tenant environments
    3.5 Limitations and Practical Considerations
        Metric availability (AMD/ARM, vendor lock-in)
        Interference and contention
        Metric resol ution and sync
        Scheduler behavior
        Measurement overhead




\section{"data fusion of power data and cpu metrics}
- Estimating the consumption of a single function has been proven to be possible in 2012: M. Hähnel, B. Döbel, M. Völp, and H. Härtig, “Measuring energy consumption for short code paths using RAPL,”, \parencite{hahnel2012measuring}

\section{Energy Attribution Techniques in containerized environments}
    introduction. the problem is non-trivial
    usage-based, event-based, statistical modelling, ...
    where to put the system usage

\subsection{Data fusion techniques}
    resource usage correlation (CPU time, Mem, I/O)
    time based attribution
    event-driven attribution

\subsection{multi-tenant complexity}
    challenges of multi-tenant workloads
    isolation issues, nosy neighbors, contention
