\chapter{Table templates}
\label{tables}


%controlled width
\begin{tabular}{ | m{5em} | m{1cm}| m{1cm} | } 
    \hline
    cell1 dummy text dummy text dummy text& cell2 & cell3 \\ 
    \hline
    cell1 dummy text dummy text dummy text & cell5 & cell6 \\ 
    \hline
    cell7 & cell8 & cell9 \\ 
    \hline
\end{tabular}

%auto width
\begin{tabular}{ |c|c|c| } 
    \hline
    cell1 dummy text dummy text dummy text& cell2 & cell3 \\ 
    \hline
    cell1 dummy text dummy text dummy text & cell5 & cell6 \\ 
    \hline
    cell7 & cell8 & cell9 \\ 
    \hline
\end{tabular}

% and with heading line

%controlled width
\begin{tabular}{ | m{5em} | m{1cm}| m{1cm} | } 
    \hline
    cell1 dummy text dummy text dummy text& cell2 & cell3 \\ 
    \Xhline{1.5pt}
    cell1 dummy text dummy text dummy text & cell5 & cell6 \\ 
    \hline
    cell7 & cell8 & cell9 \\ 
    \hline
\end{tabular}

%auto width
\begin{tabular}{ |c|c|c| } 
    \hline
    cell1 dummy text dummy text dummy text& cell2 & cell3 \\ 
    \Xhline{1.5pt}
    cell1 dummy text dummy text dummy text & cell5 & cell6 \\ 
    \hline
    cell7 & cell8 & cell9 \\ 
    \hline
\end{tabular}

\begin{tabularx}{0.8\textwidth} { 
  | >{\raggedright\arraybackslash}X 
  | >{\centering\arraybackslash}X 
  | >{\raggedleft\arraybackslash}X | }
 \hline
 item 11 & item 12 & item 13 \\
 \hline
 item 21  & item 22  & item 23  \\
\hline
\end{tabularx}
