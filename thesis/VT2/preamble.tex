% !TEX root = main.tex

%----------------------------
%   Fonts and characters
%----------------------------

% Support for special characters
\usepackage[utf8]{inputenc}    % Specify input encoding
\usepackage[T1]{fontenc}       % Specify font encoding

% Set main fonts
% Fonts catalogue: https://tug.org/FontCatalogue/
\usepackage{mathpazo}          % Use the Palatino font by default
\usepackage{beramono}          % Override the monospace/typewriter font

% ZHAW title font
% Try to load Helvetica Rounded Bold, and OpenType font.
% Loading OTF or system fonts is possible with XeLaTeX.
% If the document is compiled using pdfLaTeX, resort 
\usepackage{ifxetex}
\ifxetex
    \usepackage{fontspec}
    \newfontfamily\zhawtitlefont{Helvetica Rounded Bold}
\else
    \newcommand{\zhawtitlefont}{\scshape}
\fi

%\usepackage[scaled]{helvet}

%----------------------------
%   Environments
%----------------------------
\setcounter{secnumdepth}{3}
\usepackage{amsmath}
\usepackage{array}
\usepackage{caption}           % Customized caption
\usepackage{subcaption}        % Subfigure captions
\usepackage{makecell}          % Per-cell formatting in tables (\makecell)
\usepackage{pdfpages}          % Required to include PDF files/graphics (\includepdf)

\usepackage{todonotes}         % Introduces the command \todo
\setlength{\marginparwidth}{2.5cm} % Adjust this if the todo notes are out of margins

% Create boxes as follows:
% \begin{colorbox}{red}{2}
\usepackage{tcolorbox}
\newtcolorbox{textbox}[2]{
    arc=3pt,
    boxrule=#2pt,
    colback=#1!25!white,
    width=\textwidth,
    halign=left,
    valign=center,
    colframe=#1!75!black
}

%----------------------------
%   Colors
%----------------------------

% Set up colors
\usepackage{xcolor}
% ZHAW Blue: Pantone 2945 U / R0 G100 B166
\definecolor{zhawblue}{rgb}{0.00, 0.39, 0.65}
% Colors related to code listings
\definecolor{codegreen}{rgb}{0,0.6,0}
\definecolor{codegray}{rgb}{0.5,0.5,0.5}
\definecolor{codepurple}{rgb}{0.58,0,0.82}
\definecolor{codebackground}{rgb}{0.93,0.94,0.95}

%----------------------------
%   Code listings
%----------------------------

% Setup code listings
\usepackage{listings}
\lstdefinestyle{mystyle}{
    backgroundcolor=\color{codebackground},   
    commentstyle=\color{codegreen},
    keywordstyle=\color{magenta},
    numberstyle=\tiny\color{codegray},
    stringstyle=\color{codepurple},
    basicstyle=\ttfamily\footnotesize,
    breakatwhitespace=false,
    breaklines=true,
%    captionpos=b,
    keepspaces=true,
    numbers=left,
    numbersep=5pt,
    showspaces=false,
    showstringspaces=false,
    showtabs=false,
    tabsize=4
}
\lstset{style=mystyle}

% minted is an alternative code listing package. (See chapter 2)
% For it to run successfully, ensure the following:
% - the Python package Pygments. Install with the following command:
%       python -m pip install Pygments
% - pdflatex (or xelatex) is executed with the flag --shell-escape
%   If you are using a TEX editor, you can modify the typesetting 
%   command somewhere in the settings.
%\usepackage[outputdir=build]{minted}
%\usemintedstyle{xcode}
% For fancier coloring schemes, see here:
% https://tex.stackexchange.com/questions/585582
% One could also create an own style in Pygments
% https://pygments.org/docs/styles/#creating-own-styles

%----------------------------
%   References
%----------------------------

% Set up references
\usepackage[
    backend=biber,             % Use biber backend (an external tool)
    sorting=none,              % Enumerates the reference in order of their appearance
    style=numeric-comp         % Choose here your preferred citation style
]{biblatex}
\addbibresource{example.bib}   % The filename of the bibliography
\usepackage[autostyle=true]{csquotes} 
                               % Required to generate language-dependent quotes 
                               % in the bibliography

%----------------------------------------------------------------------------------------
%   MARGIN SETTINGS
%----------------------------------------------------------------------------------------

\geometry{
    paper=a4paper,      % Change to letterpaper for US letter
    inner=2.5cm,        % Inner margin
    outer=3.8cm,        % Outer margin
    top=1.5cm,          % Top margin
    bottom=1.5cm,       % Bottom margin
    bindingoffset=.5cm, % Binding offset
    %showframe,         % Show the type block of the page
}
\setlength{\parskip}{1em}
\usepackage{enumitem}          % Layout control for list environments (e.g, itemize)
%\setlist{noitemsep}           % Suppress extra spaces between items
%\setlist{nosep}               % Suppress spaces before/after list environments

\usepackage{float}


\immediate\write18{mkdir -p build}
\AtBeginDocument{
    \IfFileExists{./build/\jobname.aux}{\input{./build/\jobname.aux}}{}
}

\makeatletter
\def\input@path{{./build/}}
\makeatother

\usepackage{etoolbox}
\AtBeginDocument{\patchcmd{\include}{\@include}{\@input}{}{}}